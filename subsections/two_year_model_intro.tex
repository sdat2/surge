\subsubsection{NEMO Models}
\paragraph{Introduction}
I was able to use two outputs~§\ref{sec:rean-prod}~\&~§\ref{sec:control-1950-intro}
of use the European NEMO ocean engine~\cite{madec2015nemo}, which uses the ORCA12 grid, with a  $\frac{1}{12}^{\circ}$
resolution.
This is a tripolar ocean grid, designed so that no coordinate singularities are in the ocean~\cite{madec1996global}.
Instead there are two `north' poles, one in Russia, and one in the Canada,
with the grid elegantly curving between the two.
The south pole remains in Antarctica, and the grid is undistorted from a lon-lat
 grid up until 20$^{\circ}$N.\textbf{[citation needed]}

 The Bathymetry used by the ocean engine is based on
 ETOPO2~\cite{lecointre2011definition,noaa20062}.\footnote{\url{https://www.ngdc.noaa.gov/mgg/global/etopo2.html}}

\paragraph{Limitations}

Neither of the outputs in~§\ref{sec:rean-prod}~\&~§\ref{sec:control-1950-intro}
include tides, inundation, or wave setup.
The resolution of the models mean that in the region of interest cells are around
8km wide. Therefore the centre at which the sea surface height is recorded,
is around 4km from the coast.

\subsubsection{2004-5 Hourly Reanalysis Product}
\label{sec:rean-prod}
Dr.~Pierre Mathiot kindly ran a global two year ocean simulation with
hourly outputs for the years 2004-5. It used EN4 for its initial conditions,
and CORE2 for its atmospheric forcing.

\paragraph{EN4 Initial Conditions}

Initial condition EN4~\cite{good2013en4, HadObs} (start of the run = 1976).

\paragraph{CORE2 Forcing}

Atmospheric forcing came from CORE2~\cite{griffies2012datasets,large2009global}.
