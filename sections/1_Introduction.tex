\section{Introduction}
\label{sec:1_Introduction}


Storm surges are coastal sea levels which far exceed those
expected given the tide.
They are caused by either tropical cyclones
(TCs) or extratropical cyclones (ECs).
TCs draw their kinetic energy from the thermodynamic disequilibrium between
the tropical sea surface and the troposphere~\cite{emanuel1987dependence},
whereas ECs draw theirs from the
polewards temperature gradient~\cite{lorenz1960energy, holton2004introduction}.
Their respective names reflect that conditions favourable
to their formation and growth are within and without the tropics respectively.
Because they can lower central pressures and sustain greater winds,
TCs create larger and more dangerous storm surges~\cite{emanuel2005divine}.

TCs have historically been the
world's largest physical hazard~\cite{shultz2005epidemiology},
in terms of economic damage, and TC storm surges for lives lost~\cite{zhang2009tropical}.
This hazard is increasing as more people live on vulnerable coastlines
in insufficient buildings~\cite{emanuel2005divine}.
Anthropogenic climate change is expected to exacerbate this,
as it begins to both increase the maximum TC intensity\footnote{However the uptick
in the number of tropical cyclones is not currently above
 natural variability~\cite{mendelsohn2012impact}.}
(see~§~\ref{sec:hurr-theory})~\cite{emanuel2008hurricanes,emanuel2017will},
and lead to sea level rise placing more areas at risk~\cite{SROCC}.
The geographical range of hurricanes may also increase
to the ranges of previous epochs~\cite{fedorov2010tropical}.

Understanding how this hazard will change quantitatively
is a complicated question, but necessary to accurately
inform the public,
so that they can act with foresight.
Others have already begun to apply machine learning (ML) techniques to this
question~\cite{kulp2019new, kulp2018coastaldem, tadesse2020data}.



\begin{table}[ht]
\resizebox{\columnwidth}{!}{%
\begin{tabular}{l L{6.5cm}}

\hline \hline
\textbf{Ab} & \textbf{Expansion}\\
\hline
EVT & Extreme value theory \\
GEV & Generalised extreme value (distribution) \\
BM & Block maxima \\
ML & Machine learning \\
GP & Gaussian process \\
TC & Tropical cyclone \\
EC & Extratropical cyclone \\
CMIP & Coupled model intercomparison project \\
Y20 & Yin et al.~2020~\cite{ZannaPreprint} \\
T19 & Taleb 2019, \textit{The Statistical Consequences of Fat Tails}~\cite{taleb2019statistical}\\
SSH & Sea surface height above geoid (m) \\
SST & Sea surface temperature ($^{\circ}$C) \\
\texttt{tyr} & \texttt{two-year} 2004-5 hindcast (§~\ref{sec:rean-prod}) \\
\texttt{c50} & \texttt{control-1950} experiment (§~\ref{sec:control-1950-intro}) \\
\texttt{vc} & N-Vietnam \& China coastline (§~\ref{sec:coast-sample})\\
\texttt{eUS} & East US coastline (§~\ref{sec:coast-sample})\\
\hline \hline
\end{tabular}
}
\caption{Abbreviations (Ab) used in this report.}
\label{tab:abb}
\end{table}


This report begins to develop new methods of achieving this goal
as inspired by Yin~et~al.~2020~\cite{ZannaPreprint} (henceforth Y20). Initially, by
developing algorithms for extracting the coastlines from climate models
and metrics to characterise that coastline in §~\ref{sec:coast}.
Developing from Y20~\cite{ZannaPreprint} I compare measures for the
responsiveness of a stretch of the coastline to a certain wind stress
level. I show using extreme value theory (EVT), that the
distributions of extreme events are controlled by the same factors as
the responsiveness under normal circumstances. Finally, I suggest a
novel method of improving a naive EVT fit, using our physical knowledge
that there is a thermodynamical limit to the maximum intensity of a hurricane~\cite{emanuel1999thermodynamic}
 given a climate (see~§~\ref{sec:hurr-theory})~\cite{emanuel1987dependence}, albeit with some level of
 uncertainty~\cite{emanuel2016predictability}.


I discuss the merits and deficiencies of these new investigations (§~\ref{sec:future}),
and suggest how they could be built upon in future work (§~\ref{sec:sys-errors}).
