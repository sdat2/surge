\begin{figure*}[htb!]
    \centering
    \vspace{-5pt}
    \includegraphics[width=0.8\linewidth]{../surge/plots/GEV_modelNO.pdf}
    \vspace{-15pt}
   \caption{New Orleans GEV plot adapted from \texttt{skextremes}~\cite{skextremes}
            for \texttt{c50} with $\alpha=0.05$ . Interesting transition
            - does this represent hurricanes? (see Figure~\ref{fig:return_hyp_new})
            The deviation from the fitted distribution
            is shown in A~\&~C. The parameters above are quoted with 1$\sigma$ error bars.
            C's error bars are 2$\sigma$.}
    \label{fig:gev-no}
    \includegraphics[width=0.8\linewidth]{../surge/plots/skextreme_second_tactic.pdf}
    \vspace{-15pt}
   \caption{GEV parameters for \texttt{c50}, \texttt{eUS}.
   The confidence intervals, particularly in $\xi$, show that the fit of the
   GEV to the data is relatively poor. $\xi$ is more likely to be negative
   in the Gulf of Mexico, which could because of higher
   TC activity along this coastline (Figure~\ref{fig:top-100}).}
    \label{fig:gev_all_points}
\end{figure*}
