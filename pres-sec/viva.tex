
\begin{frame}{Introduction}

\vspace{-20pt}
\paragraph{Based on four key sources:}
\begin{itemize}
\item Mandelbrot 1967 (M67,~\cite{mandelbrot1967long}) -- Coasts, Fractals.
\item Emanuel 1986 (E86,~\cite{emanuel1986air}) -- Tropical Cyclones (TCs).
\item Taleb 2019 (T19,~\cite{taleb2019statistical}) -- Extreme Value Theory (EVT).
\item Yin et al.~2020 (Y20,~\cite{ZannaPreprint}) -- Inspiration.
\end{itemize}
\paragraph{Storm surges are created by ( w.o. Tide-surge interaction):}
\begin{itemize}
\item  Wind stress creates $\sim 80$\% of a surge. In steady state\footnote{from Pugh, 1987~\cite{pugh1987tides} p.~89~\&~199.},
\begin{equation}
\frac{\partial \eta}{\partial x}
\approx \frac{\tau}{\rho_{0} g H}.
\label{eq:pugh}
\end{equation}
\item Wave buildup $\sim 15$\% (not captured).
\item Pressure pull-up $\sim 5$\% (not recorded).
\end{itemize}
\paragraph{This report uses:}
\begin{itemize}
\item \texttt{surge}~\cite{gitlab}, \texttt{skextremes}~\cite{skextremes},
      \texttt{sklearn}~\cite{scikit-learn}, \& \texttt{xarray}~\cite{hoyer2017xarray}.
\item \texttt{two-year} `04-`05 (\texttt{tyr}), \&  \texttt{control-1950} 101-yrs (\texttt{c50}).
\end{itemize}

\end{frame}



\begin{frame}{}

\begin{minipage}{0.45\linewidth}

\includegraphics[width=\linewidth]{../surge/plots/angle_heatmap.pdf}\\
$\uparrow$~Bearing \& $\downarrow$~Convexity \texttt{eUS}.

\includegraphics[width=\linewidth]{../surge/plots/derivative_heatmap.pdf}

\end{minipage} \begin{minipage}{0.45\linewidth}
\raggedleft
\includegraphics[width=\linewidth]{../surge/plots/bath_list.pdf}\\
{$\uparrow$ Isobaths on \texttt{eUS}.}

{$\downarrow$ Distance from \texttt{eUS}.}
\includegraphics[width=\linewidth]{../surge/plots/distance_isobath.pdf}\\

\end{minipage}

\end{frame}

\begin{frame}
%\hspace{-40pt}
\begin{minipage}{0.6\linewidth}
\includegraphics[width=\linewidth]{../surge/plots/rmlr.pdf}
 %\vspace{-15pt}

\label{fig:tau-tau-resp}
\end{minipage}
%\hspace{-40pt}
\begin{minipage}{0.3\linewidth}
\raggedright
{$\leftarrow$ Regressing
     $\Delta\eta_{\mathrm{hp}}$ to $\tau_u$ and $\tau_v$.}
\hspace{50pt}

\raggedleft
\includegraphics[width=\linewidth]{../surge/plots/reg_angle.png}\\
{$\uparrow$ Normal bearing and regression line similar.}
\end{minipage}

 \label{fig:tau-tau-angle}
 \begin{minipage}{0.6\linewidth}
 \includegraphics[width=\linewidth]{../surge/plots/adj_reg_mag.pdf}
 \end{minipage}
 \begin{minipage}{0.3\linewidth}
 {$\leftarrow$ Responsiveness magnitude.}
 \end{minipage}
\end{frame}

\begin{frame}
\centering \vspace{-20pt}
\begin{align}
    \operatorname{GEV}(x; \mu, \sigma, \xi)&=&
    \frac{1}{\sigma} \chi(x)^{1-\xi} e^{-\chi(x)}; \tag{GEV-1} \label{eq:GEV-1} \\
    \chi(x)&=&\left\{\begin{array}{ll}
    \left(1-\xi\left(\frac{x-\mu}{\sigma}\right)\right)^{1 / \xi} & \text { if } \xi \neq 0 \\
    e^{-(x-\mu) / \sigma} & \text { if } \xi=0 \tag{GEV-2}
    \end{array}\right.
   \label{eq:GEV-2}
\end{align}

\includegraphics[width=0.8\linewidth]{../surge/plots/GEV_modelNO.pdf}\\
NO GEV plot \texttt{skextremes}~\cite{skextremes}
        for \texttt{c50}.
         A~\&~C show bad fit?
        %Parameters with 1$\sigma$ error.
        %C's error bars are 2$\sigma$.

\end{frame}

\begin{frame}{\texttt{c50}, \texttt{eUS}, GEV parameters}
\vspace{-20pt}

\includegraphics[width=0.8\linewidth]{../surge/plots/skextreme_second_tactic.pdf}\\
CI of GEV show that fit is poor.
$\xi$ is more negative
in Gulf of Mexico. $r_p$
show that $\mu$ and $\sigma$ are correlated (all $r_p$ have 1\%$\gg$p).
$\mu$ and responsiveness: $r_p=0.44\pm0.02$, p$<10^{-30}$.
\end{frame}

\begin{frame}{The Potential Intensity of Tropical Cyclones (TCs)}
\vspace{-30pt}
\hspace{-30pt}\begin{minipage}{1.1\linewidth}
\centering
\begin{minipage}{0.45\linewidth}
\centering
    \includegraphics[width=\linewidth]{images/hurricane-carnot.png}\\
    \textit{Figure 1 from~\cite{emanuel1991theory}.
    TCs are a Carnot cycle. }
    \end{minipage}
\begin{minipage}{0.45\linewidth}
\includegraphics[width=\linewidth]{kat-heat.pdf}\\
\textit{Daily downwards heat flux
        \texttt{tyr}.}
       \end{minipage}
\end{minipage}
\begin{itemize}
\item TCs are a finite amplitude wind induced heat exchange (WISHE) instability.
\item The potential intensity (equation 15-7 in \cite{emanuel2018progress}) is,

\begin{minipage}{0.45\linewidth}
\begin{equation}
\left|\mathbf{V}_{s}\right|^{2}=\frac{C_{k}}{C_{D}}
\frac{T_{s}-T_{o}}{T_{o}}\left(k_{0}^{*}-k\right),
\tag{PI}
\label{eq:PI}
\end{equation}
\end{minipage}
\begin{minipage}{0.45\linewidth}
\begin{equation}
k \equiv c_{p} T+L_{v} q,
\label{eq:enthalpy_per_unit_mass}
\end{equation}
\end{minipage}
\end{itemize}
\end{frame}

\begin{frame}{%Yearly PI $\implies$ BM plateau.
}
\centering
\vspace{-1pt}
 \begin{minipage}{0.45\textwidth}
    \centering
    \includegraphics[width=1\linewidth]{images/taleb-limit-slimmed.png}\\
    \textit{Figure 15.1 from T19~\cite{taleb2019statistical} p.~279}
   \end{minipage} \begin{minipage}{0.45\textwidth}
   \includegraphics[width=1\linewidth]{../surge/plots/GEV_pi_plateau_NO.pdf}
   Attempt at enforcing a GP asymptote of 2m for NO.
   %1$\sigma$ and 2$\sigma$ envelopes shown.
   %\label{fig:gp-plateau}
   \end{minipage}
\includegraphics[width=0.7\linewidth]{images/PI-max-year.png}\\
\textit{Figure 15-7 from \cite{emanuel2018progress}.}
Annual maximum of the PI (m s$^{-1}$), calculated using~\cite{bister2002low}
and ERA-Interim data 1979-2016~\cite{dee2011era, berrisford2009era}.

\end{frame}


\begin{frame}{Summary}
\centering
\includegraphics[height=2cm]{images/NASA-KATRINA-SIDEON.jpg}
\begin{itemize}
\item Extract the coastline, and characterise it quantitatively.
\item Show that the responsiveness behaves as expected.
\item Use extreme value theory to estimate hazard.
\item Make risk tractable through physical constraints.
\item (Oceanography looks like CM-physics).
\end{itemize}
\centering
\includegraphics[height=3cm]{../surge/plots/theory.pdf}

\end{frame}
