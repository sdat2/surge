\subsection{Surge buildup}

The majority of the surge is caused by the wind
sustaining a stress sustained slope,
rather than the low-pressure directly
(see Y20~\cite{ZannaPreprint} Figure 8)~\cite{emanuel2005divine}. Equation 10 in Y20~\cite{ZannaPreprint}
(from Pugh, 1987~\cite{pugh1987tides} p.~89~\&~199) states that in a steady state box model,

\begin{equation}
\frac{\partial \eta}{\partial x}
\approx \frac{\tau}{\rho_{0} g H},
\label{eq:pugh}
\end{equation}

where $\eta$ is the dynamical sea level, $\rho_0$ is the seawater density,
$g$ is the gravitational acceleration, and $H$ is the ocean depth.
In this model
the storm surge
will grow linearly with fetch,
quadratically with wind speed (linearly in \ref{eq:PI}),
and inversely with the depth.
This implies that the surge buildup is greater if there is an area of
shallow water by the coast, there will
be a greater risk of storm surges.


\paragraph{Wave buildup}

A gradual slope will also favour the buildup of waves in the same location, due to
improved impedance matching, with more of the wave energy being able to
reach the shore~\cite{pugh1987tides}. However those areas with shallowly sloping
bathymetrys may be protected by headlands.
