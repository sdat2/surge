\subsubsection{NEMO models}
\paragraph{Introduction}
\label{sec:nemo}
I was able to use two outputs~§~\ref{sec:rean-prod}~\&~§~\ref{sec:control-1950-intro}
of use the NEMO ocean engine~\cite{madec2015nemo}, which uses the tripolar ORCA12 grid,
with a  $\frac{1}{12}^{\circ}$
resolution, designed so that no coordinate singularities are in the ocean~\cite{madec1996global}.


 The bathymetry is from
 ETOPO2v2~\cite{lecointre2011definition, noaa20062}.\footnote{\url{https://www.ngdc.noaa.gov/mgg/global/etopo2.html}}
 This is more accurate along the US coastline, using
 direct measurements rather than altimetry data~\cite{noaa20062}.
 Table~\ref{tab:mod-symb} introduces the variables taken from NEMO.




\begin{table}[ht]
\resizebox{\columnwidth}{!}{%
\begin{tabular}{l L{5.5cm} l}

\hline \hline
\textbf{Sym} & \textbf{Name} & \texttt{cn}\\
\hline
$\eta$ & SSH & \texttt{zos} \\
$\tau_u$ & Sea surface stress along u (Pa) & \texttt{tauuo} \\
$\tau_v$ & Sea surface stress along v (Pa)& \texttt{tauvo} \\
$|U_{10}|$ & 10m windspeed (m s$^{-1}$) & \texttt{sowindsp}\\
$T_s$ & Sea surface temperature (SST) & \texttt{tos} \\
$q_s$ & Downwards heat flux & \texttt{hfds}\\
\hline \hline
\end{tabular}
}
\caption{Symbols (sym), names and \texttt{code-names} (\texttt{cn})
taken from NEMO outputs in this report.}
\label{tab:mod-symb}
\end{table}


\paragraph{Limitations}

Neither output
include tides, inundation, or wave setup.
The resolution means that cells are around
8km wide, therefore the sea surface height (SSH)
is recorded around 4km from the coast.

\subsubsection{2004-5 \texttt{two-year} hourly reanalysis product}
\label{sec:rean-prod}
Dr.~Pierre Mathiot kindly ran a global ocean model with
hourly outputs for 2004-5 (\texttt{tyr}) (see Figure~\ref{fig:katrina}).

\paragraph{EN4 initial conditions}
EN4 is a reanalysis product, starting in 1976,
 that uses all available ocean profiles~\cite{good2013en4, HadObs}
to produce a best guess as to what the properties of the ocean are.


\paragraph{CORE2 forcing}
CORE2 is a set of global air-sea fluxes in 6 hour time steps which are interpolated between~\cite{griffies2012datasets,large2009global,
 hurrell2008new}\footnote{\url{https://climatedataguide.ucar.edu/climate-data/corev2-air-sea-surface-fluxes}}
 (see Figure~\ref{fig:katrina}B-D).
 These were made from the a reanalysis product,
 bias corrected with satellite data~\cite{core2, core2expert}.


\begin{figure}[htb!]
\centering
\includegraphics[width=\linewidth]{../surge/plots/katrina_graph.pdf}\\
\vspace{-9pt}
\includegraphics[width=\linewidth]{../surge/plots/new_orleans_map.pdf}
\caption{The representation of the landfall of Hurricane Katrina within the
two year analysis product as seen from three selected points along
the coast around New Orleans (NO). As shown in D, the eye of the TC
crosses passes through this area. The wind stresses experienced at the
the three points are similar (B~\&~C), but the response (A) is much
less significant at the headland than the other two points.
B-D are consistent with CORE2's fields being interpolated from 6~hour
steps as in~\cite{core2expert}.
}
\label{fig:katrina}
\end{figure}

