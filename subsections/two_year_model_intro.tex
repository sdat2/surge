\subsubsection{NEMO models}
\paragraph{Introduction}
\label{sec:nemo}
I was able to use two outputs~§~\ref{sec:rean-prod}~\&~§~\ref{sec:control-1950-intro}
of use the European NEMO ocean engine~\cite{madec2015nemo}, which uses the ORCA12 grid, with a  $\frac{1}{12}^{\circ}$
resolution.
This is a tripolar ocean grid, designed so that no coordinate singularities are in the ocean~\cite{madec1996global}.
Instead there are two `north' poles, in Russia and Canada,
with the grid elegantly curving between the two.
The grid is undistorted from a lon-lat
 grid in the southern hemisphere.

 The bathymetry used by the ocean engine is based on
 ETOPO2~\cite{lecointre2011definition, noaa20062}.\footnote{\url{https://www.ngdc.noaa.gov/mgg/global/etopo2.html}}
 This is more accurate along the US coastline as it uses
 direct measurements rather than relying on altimetry data~\cite{noaa20062}.
 Table~\ref{tab:mod-symb} introduces the variables taken from NEMO.




\begin{table}[ht]
\resizebox{\columnwidth}{!}{%
\begin{tabular}{l L{5.5cm} l}

\hline \hline
\textbf{Sym} & \textbf{Name} & \texttt{cn}\\
\hline
$\eta$ & SSH & \texttt{zos} \\
$\tau_u$ & Sea surface stress along u (Pa) & \texttt{tauuo} \\
$\tau_v$ & Sea surface stress along v (Pa)& \texttt{tauvo} \\
$|U|$ & Surface Ocean Windspeed (m s$^{-1}$) & \texttt{sowindsp}\\
\hline \hline
\end{tabular}
}
\caption{Symbols (sym), names and \texttt{code-names} (\texttt{cn})
taken from NEMO outputs in this report.}
\label{tab:mod-symb}
\end{table}


\paragraph{Limitations}

Neither of the outputs in~§~\ref{sec:rean-prod}~\&~§~\ref{sec:control-1950-intro}
include tides, inundation (the flooding of land), or wave setup.
The resolution of the models mean that in the region of interest cells are around
8km wide. Therefore the centre at which the sea surface height is recorded,
is around 4km from the coast.

\subsubsection{2004-5 \texttt{two-year} hourly reanalysis product}
\label{sec:rean-prod}
Dr.~Pierre Mathiot kindly ran a global two year ocean simulation with
hourly outputs for the years 2004-5 (thus including a representation of Katrina as
in Figure~\ref{fig:katrina}).

\paragraph{EN4 initial conditions}
EN4 is a product that uses all available ocean profiles
(through the ARGO program etc.)~\cite{good2013en4, HadObs}
to produce a best guess as to what the salinity and temperature of the ocean is.
 The start of this reanalysis product is 1976.


\paragraph{CORE2 forcing}
CORE2 is a set of global air-sea fluxes~\cite{griffies2012datasets,large2009global,
 hurrell2008new}.\footnote{\url{https://climatedataguide.ucar.edu/climate-data/corev2-air-sea-surface-fluxes}}
 The primary data source is the NCEP/NCAR R1 reanalysis product,
 but bias corrected with more reliable remote and local sensing data~\cite{core2, core2expert}.
 The CORE2 output is produced in 6 hour time inputs which are interpolated
 (see Figure~\ref{fig:katrina}B-D).


\begin{figure}[htb!]
\centering
\includegraphics[width=\linewidth]{../surge/plots/katrina_graph.pdf}\\
\vspace{-9pt}
\includegraphics[width=\linewidth]{../surge/plots/new_orleans_map.pdf}
\caption{The landfall of Hurricane Katrina within \texttt{tyr}  from three points
 around New Orleans (NO). As shown in D, the eye of the TC
crosses passes through this area. The wind stresses experienced at the
the three points are similar (B~\&~C), but the response (A) is much
less significant at the headland than the other two points.
B-D are consistent with CORE2's fields being interpolated from 6~hour
steps as in~\cite{core2expert}.
The bearing angles are explained in §~\ref{sec:convexity}.
}
\label{fig:katrina}
\end{figure}

