\section{Extreme value theory}
\label{sec:5_Transform}
\subsection{Idealised statistical theory}
The central limit theorem is a well known triumph of classical statistics,
and it is often useful in the physical sciences where the central value
is that of interest. However, by its nature the mean SSH will
not be able to damage the coast, and life and property will only be put
at risk by the most extreme events.

Extreme value theory is the provides the extension of this theory
to look at the return period of extreme events of a certain value.

Both require domain of attraction condition~\cite{bucher2018horse}:

\begin{enumerate}
  \item Either $\forall r \in \mathbb{N} \;\exists \;b_r \in \mathbb{R},\; \gamma\in \mathbb{R},\; a_r>0: $
    \begin{eqnarray}
    \lim _{r \rightarrow \infty} F^{r}\left(a_{r} x+b_{r}\right)=\exp \left\{-(1+\gamma x)^{-1 / \gamma}\right\} \\
     \text { for all } 1+\gamma x>0
    \tag{BM}
    \end{eqnarray}

  \item Or equivalently, there exists a positive function $\psi=\psi (t):$
    \begin{eqnarray}
    \lim _{t \uparrow x^{*}} \frac{1-F(t+\psi(t) x)}{1-F(t)}=(1+\gamma x)^{-1 / \gamma} \\
    \text { for all } 1+\gamma x>0
    \tag{POT}
    \end{eqnarray}
 \end{enumerate}

Normal Choices are POT or Block Maxima, but these strategies require
large amounts of data to converge~\cite{taleb2019much}.

BM leads to Generalised Extreme Value (GEV) distribution,
      whereas POT leads to Generalised Pareto Distribution (GPD).

\subsection{Block maxima GEV}
It is trivial to extract the highest SSH value in a given year at a given point
from \texttt{control-1950}.
It was decided not to subtract the low frequency SSH seasonal cycle from the
values first, because it is the absolute height of the sea surface which
creates the hazard rather than the relative height.

\subsection{Comparison to POT GPD}
The other alternative is to use the peak-over-threshold method.
There are a number of alternatives as to how you should choose the
threshold.

\subsection{Enforcing an asymptote using potential intensity theory }
As noted in~§~\ref{sec:hurr-theory} there is some maximum size
that a tropical cyclone might be expected to be able to reach given
the climate. This gives us information as to the shape of the probability
distribution beyond what just curve-fitting the tail.
