\section{Simple measures of responsiveness}
\label{sec:3_ORCA12_REG.tex}
\subsection{Linear ML models}
\label{sec:lin-ml-models}

To regress some set of inputs, $X$, against
some set of targets, $Y$,
we need to choose an objective function to minimise:

\begin{align}
J_{\mathrm{mlr}} = & \|X w-y\|_{2}^{2} \tag{MLR}, \label{eq:MLR} \\
J_{\mathrm{lasso}} = &
\frac{1}{2 n_{\text {samples }}}\|X w-y\|_{2}^{2}+\alpha\|w\|_{1} \tag{LAS}, \label{eq:LAS} \\
J_{\mathrm{ridge}} = &  \|X w-y\|_{2}^{2}+\alpha\|w\|_{2}^{2} \tag{RID}, \label{eq:RID}
\end{align}

Where \ref{eq:MLR} is the ordinary minimisation of least squares, that has no
penalty for the complexity of parameters.
The lasso objective function adds a penalty~\ref{eq:LAS},
$\alpha\|w\|_{1}$ added, where $\alpha$ is a constant and $\|w\|_{1}$ is the
$l_1$-norm of the coefficient vector.
The ridge objective function~\ref{eq:RID}
$\alpha\ge0$ controls the amount of shrinkage:
the larger value of $\alpha$,
the greater the amount of shrinkage
and thus the coefficients become more robust to multi-colinearity.
A coordinate descent algorithm is used to~\cite{scikit-learn}
$
\min _{w} (J)
$.

The complexity parameter $\alpha$, can be set to
the value that leads to the greatest generalisability to unseen data.


As shown in Figure~19~of~T20~\cite{ZannaPreprint}.
The threshold could be arbitrary,
so I chose that the SSH has to be greater than zero,
as it approximately guarantees that the sea surface stress is acting into
the coast rather than away from it.


\subsection{$\tau_u$, $\tau_v$ responsiveness}
\label{sec:tau-tau}
This method is quite crude, and could be simply improved.
Figure~\ref{fig:tau-tau-r-no} shows a single example of working out the linear
responsiveness (or linear predictability).
Figure~\ref{fig:tau-tau-resp} shows that when this same metric is applied along \texttt{eUS}
for each year in \texttt{tyr}. Figure~\ref{fig:tau-tau-responsiveness} creates a metric for the
responsiveness.



\begin{figure*}
\centering
 \hspace{-40pt} \includegraphics[width=0.6\linewidth]{../surge/plots/rmlr.pdf}
  \vspace{-15pt}
 \caption{A four panel plot to show the estimation of responsiveness
          of $\Delta\eta_{\mathrm{hp}}$ to $\tau_u$ and $\tau_v$, with the
          \texttt{MLR} and \texttt{Huber} algorithms. The y-units of A~\&~B are
           m Pa$^{-1}$.}
 \label{fig:tau-tau-resp}
 \hspace{-40pt} \includegraphics[width=0.3\linewidth]{../surge/plots/reg_angle.png}
  \vspace{-15pt}
 \caption{This shows that for the majority of the coastline there is a close
 corresponce between the normal bearing of the coast and the regression line ($r_p=0.83\pm0.05$).
 \texttt{np.arctan2(c0, c1)}}
  \label{fig:tau-tau-angle}
  \hspace{-40pt} \includegraphics[width=0.5\linewidth]{../surge/plots/adj_reg_mag.pdf}
   \vspace{-15pt}
  \caption{A more robust measure of responsiveness magnitude.
  ($\bar{r^2}$)\texttt{*np.sqrt(np.square(c0) + np.square(c1))}. }
   \label{fig:tau-tau-responsiveness}
\end{figure*}


\label{sec:angle}

Figure~\ref{fig:tau-tau-angle} shows that the angle is highly correlated with
the bearing of the coast that was defined in §~\ref{sec:convexity}.
The bearing of the regression line can be simply calculated,
and compared to the bearing of the normal coastline.
As shown, there is a systematic offset between the two lines,
which changes between them. This can be explained through the
classic theory of Ekman transport of the water.
Ekman~\cite{hope2013hindcast}.


\label{sec:reg-metrics}

The magnitude of this regression line varies in approximately the same
way for both methods, showing that there is some underlying property of the
coastline that both illuminate. In the previous section, I outlined
a series of metrics.

\label{sec:generalisability}

It would be possible that this is a spurious structure, where the fit
is made possible by giving so many information-less channels to fit against.
Therefore we need to use the other coasts to compare this against.
We use the Japanese and Chinese coast.

\begin{figure}[htb!]
    \centering
    \includegraphics[width=0.7\linewidth]{../surge/plots/ridge_lasso.pdf}
    \vspace{-15pt}
   \caption{\texttt{eUS} regression learning.
    Although not for lasso, and the regression is not consistent.
            We also need another coast to see if this generalises.}
    \label{fig:learnt-eus}

    \centering
    \includegraphics[width=0.8\linewidth]{../surge/plots/vc_ridge_lasso.pdf}
    \caption{The same as above, but for the \texttt{vc} coastline.}
    \label{fig:learnt-vc}
\end{figure}




\FloatBarrier
