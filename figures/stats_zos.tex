
\begin{figure*}[htb!]
    \centering
    \includegraphics[width=0.6\linewidth]{../surge/plots/stats_points_plot.pdf}
       \hspace{0pt} \includegraphics[width=0.285\linewidth]{../surge/plots/stats_points_plot_1.pdf}
    \vspace{-7pt}
    \caption{A comparison between the distributions of sea surface height, $\eta$ (SSH/zos).
     above geoid for 2005 (red) and 2004 (blue).
     Kurtosis in frame D is the excess above that expected for a normal distribution,
     which shows that areas like NO are highly non-normal (a statistical test
     shows that the probability of it being sampled from a Gaussian is $<10^{-304}$)~\cite{anscombe1983distribution}.
     One anomaly in the data is that the mean of all of the points for the coastline
     seems to be uniformly shifted upwards between the two years (A~\&~E),
     where the difference is $0.06\pm0.01$m.
      Looking at the individual points' SSH (Figure~\ref{fig:individual_zos} in §~\ref{sec:sum-var-stat}),
      does not reveal any discontinuity in the data in the two year period,
       but Figures~\ref{fig:four-trans-panels}-\ref{fig:mm-hp-lp}
      suggest that this might be caused by model spin up from the
      EN4 initial conditions using the CORE2 forcing.
      On the right of the plots (A--D) is the Pearson regression coefficient $r_p$,
      between the two years for that moment, which shows that the mean and
      standard deviation of a point are relatively static between the years, but the
      higher order moments of skew and kurtosis (C~\&~D) are more sensitive to the
      random incidence of storms along different areas of the coast.}
   \label{fig:ssh_stats_america}
\end{figure*}
