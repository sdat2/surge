\section{Introduction}
\label{sec:1_Introduction}

Machine Learning\footnote{The use of the term Machine Learning (ML) is unfortunate,
I would prefer it if ``Learning Algorithms'' or ``Advanced Statistical Modelling''
 were used by the field as I find these more descriptive.
  Perhaps due to the pressures research grant proposals and IPOs,
   the field tends to use the quite cryptic labels `AI' and `ML',
   and I will do likewise for consistency.
   I here define define Machine Learning as any algorithm applied to data
    to extract information which is more advanced than classical statistics.
    This may seem hopelessly broad, but I have not heard of one better. }
algorithms are often directly inspired by physical processes (e.g.\
stochastic differential equations (SDEs)) and concepts (e.g.\ transfer entropy).
In the mathematics that follows, the reader may be strongly reminded of Quantum Mechanics.
This is no accident as both this and ML ultimately boil down to linear algebra and vector calculus.
Indeed the two may enjoy a combined future with the introduction of quantum computation~\cite{biamonte2017quantum}.

This work follows in the footsteps of the late Professor Sir David MacKay
in two respects: in its focus on the using a Bayesian framework for inference~\cite{ITILA, MacKay91},
and in its ultimate goal to provide clear information about climate~\cite{mackay2008sustainable}.

I should note that because of the rotation-induced symmetry of western ocean boundaries~\cite{hogg1995western},
 any model that works here will generalise better to East Asia (Vietnam, Philippines,
  and China) than it would to the United Kingdom.

\subsection{Basic Oceanographic Equations}
The Coriolis parameter $f$ is given by
\begin{equation}
f=2 \Omega \sin \varphi,
\label{eq:coriolis}
\end{equation}
where $\varphi$ is the latitude of the position and $\Omega$ is the angular
 frequency of the Earth's rotation.

Fluid mechanics is based on three simple equations~\cite{landau1959course}:
 The conservation of matter,
\begin{equation}
\frac{\partial \rho}{\partial t} + \boldsymbol{\nabla}(\rho \boldsymbol{u}) = 0,
\tag{Matter}
\label{eq:Matter}
\end{equation}
the conservation of momentum,
\begin{equation}
\rho\left (\frac{\partial \mathbf{u}}{\partial t}
+ \mathbf{u}(\boldsymbol{\nabla}\cdot \mathbf{u})\right) =
- \boldsymbol{\nabla}P + \boldsymbol{\nabla}\cdot \underline{\underline{\tau}}
- \rho\boldsymbol{\nabla}\phi,
\tag{Momentum}
\end{equation}
and the conservation of energy,
\begin{equation}
\begin{array}{l}
\rho \left( \frac{\partial h}{\partial t} + \boldsymbol{\nabla}
\cdot (h\mathbf{u})\right) = \\ \quad\quad- \left( \frac{\partial P}{\partial t}
+ \mathbf{u}\cdot (\boldsymbol{\nabla} P) \right)
+ \boldsymbol{\nabla}\cdot (k \boldsymbol{\nabla} T )
+ (\underline{\underline{\tau}}\cdot\boldsymbol{\nabla})\mathbf{u}.\end{array}
\tag{Energy}
\end{equation}
Where we have assumed that we are in an inertial frame and the variables are
labelled as in Table~\ref{tab:fluid_variables}.

\begin{table}[h!]
    \centering
    \begin{tabular}{ll}
        $\rho$ & Density. \\
        $t$ & Time. \\
        $\mathbf{u}$ & Velocity. \\
        $P$ & Pressure. \\
        $\underline{\underline{\tau}}$ & Deviatronic stress (2nd order tensor). \\
        $\phi$ & Gravitational potential. \\
         $h$ & Enthalpy.\\
         $T$ & Temperature. \\
         $k$ & Thermal conductivity.\\
         $c_p$ & Heat capacity.\\
         $\eta$ & Viscosity. \\
         $H$ & Heat production density.\\
    \end{tabular}
    \caption{Symbols for fluid mechanical equations}
    \label{tab:fluid_variables}
\end{table}
We can also normally assume that all density variations can be ignored that
 are not on the gravitational term (the Boussenesq approximation),
  which simplifies the equations considerably as Equation~\ref{eq:Matter} becomes
\begin{equation}
    \boldsymbol{\nabla}\cdot\mathbf{u} = 0, \tag{B-Matter}
\end{equation}
and we also could derive that
\begin{equation}
\rho\left (\frac{\partial \mathbf{u}}{\partial t}
+ \mathbf{u}(\boldsymbol{\nabla}\cdot \mathbf{u})\right)=
-\nabla P+\eta \nabla^{2} \mathbf{u}-\rho \boldsymbol{\nabla}\phi, \tag{B-Momentum}
\end{equation}
\begin{equation}
\rho c_{p}\left(\frac{\partial T}{\partial t}
+\mathbf{u} \cdot \nabla T\right)=k \nabla^{2} T+\rho H. \tag{B-Energy}
\end{equation}
However the surface of the Earth reference frames rotate,
 which makes these equations less elegant.
  The final form of the equations in an earth bound reference frame
  is found by applying the identity
\begin{equation}
\frac{\mathrm{d}}{\mathrm{d} t} \boldsymbol{A}=
\left[\left(\frac{\mathrm{d}}{\mathrm{d} t}\right)_{r}
+\mathbf{\Omega} \times\right] \boldsymbol{A},
\end{equation}
for converting to a rotating reference frame to the previous equations.
\begin{equation}
\begin{array}{l}
\rho \frac{D \mathbf{u}}{D t}=\\\quad\quad-\boldsymbol{\nabla} P+
\eta \nabla^{2} \mathbf{u}+\frac{1}{3} \eta \nabla(\nabla \cdot \mathbf{u})
+\rho \mathbf{g}\\\quad\quad-\rho\left(2 \mathbf{\Omega} \times \mathbf{u}
+\mathbf{\Omega} \times(\mathbf{\Omega} \times \mathbf{x})
+\frac{d \mathbf{u}}{d t}\right)\end{array}.
\tag{R-Momentum}
\end{equation}



\section{Simple Tropical Cyclone Models}

One of the dominant controls on storm surge events on the Eastern Seaboard
 of the US are the large tropical cyclones (Hurricanes).
  These large vortices can be surprisingly well modelled by azimuthally
   symmetric analytic pressure distributions, and the velocity distribution
    that would be expected to result from them.

The most prominent is the  Holland model~\cite{holland1980analytic,holland2010revised}
 which elaborated on a class of functions initially proposed by Schloemer 1954~\cite{schloemer1954analysis}.
  This model uses the dimensionless pressure \( P^{\prime}\) given by

\begin{equation}
    P^{\prime}=\dfrac{P-P_{c}}{P_{n}-P_{c}},
\end{equation}
where $P_c$ is the pressure at the centre of the storm, and $P_n$ is
the ambient pressure, ideally at at an infinite distance from the storm.
They then define the pressure relation
\begin{equation}
    r^{B}\ln(\frac{1}{P^{\prime}})=A,
\end{equation}or in a more elegant form
\begin{equation}
P^{\prime}=\exp{(-\frac{A}{r^B})},
\end{equation}
where A and B are model parameters which must be found empirically.

In steady state around this  pressure distribution generally yields a
 velocity~\cite{roisin2010GFD} of:
\begin{equation}
    U_g = \sqrt{AB(P_{n} - P_{c})\frac{\exp{(-\frac{A}{r^{B}})}}{\rho r^{B}}
     +\frac{r^2f^2}{4}}-\frac{rf}{2}.
\end{equation}
If we assume that the eye of the TC itself is small s.t. the Coriolis force
 is much smaller than the centrifugal then we can approximate this as

\begin{equation}
U_c = \sqrt{AB(P_{n}-P_{c})\frac{\exp{(-A/r^{B})}}{\rho r^{B}}}.
\end{equation}
The maximum velocity
\begin{equation}
U_m = (\frac{B}{\rho e})^{0.5}(P_n-P_c)^{0.5},
\end{equation}
which occurs at
\begin{equation}
R_m = A^{1/B}.
\end{equation}
There are at least two degrees of freedom:
B defines the height of the peak and A determines its relative position.
 The difference between the ambient and central pressure of the vortex still
 enter into the velocity expression, so there are four degrees of freedom when
  those are included.


\subsection{The Rankine Vortex}

The Rankine vortex is
\begin{equation}
C=\left\{\begin{array}{ll}U \frac{R_m}{ r}& \text{ if } r < R_m,\\
U\frac{r}{ R_m} & {\text { if } r>R_m},
\end{array}\right.
\end{equation}
where C is a constant, and the two sections are joined together
 so that the velocity is continuous at the boundary.
   The two degrees of freedom are $R_m$ and C.
This can be modified to a more general relation where $Ur^{x}=\text{constant}$
where X is a new parameter to fit, which lies between 0.4 and 0.6~\cite{holland1980analytic}.



\section{Stochastic Differential Equations}

\cite{mazo2002brownian}\cite{solin2016stochastic}

\begin{equation}
\kappa(\tau)=\left\{\begin{array}{ll}
{\mathbf{H} \mathbf{P}_{\infty} \mathbf{\Phi}(\tau)^{\top
 \mathbf{H}^{\top},} & {\text { if } \tau \geq 0} \\
{\mathbf{H} \mathbf{\Phi}(-\tau) \mathbf{P}_{\infty}
 \mathbf{H}^{\top},} & {\text { if } \tau<0}
\end{array}\right.
\end{equation}
% \printbibliography
% \end{document}
