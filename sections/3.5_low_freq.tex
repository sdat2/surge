\section{Preprocessing coastal data}
\subsection{Fourier Transform}
\label{sec:fourier}

\begin{figure*}[htb!]
    \centering
    \includegraphics[width=0.6\linewidth]{../surge/plots/stats_points_plot.pdf}
       \hspace{0pt} \includegraphics[width=0.285\linewidth]{../surge/plots/stats_points_plot_1.pdf}
    \vspace{-7pt}
    \caption{A comparison between the distributions of sea surface height, $\eta$ (SSH/zos).
     above geoid for 2005 (red) and 2004 (blue),
     with the Pearson correlation coefficient, $r_p$, for each moment between the two years in A--D.
     Kurtosis in frame D is the excess above that expected for a normal distribution,
     which shows that areas like NO are highly non-normal (a statistical test
     shows that the probability of it being sampled from a Gaussian is $<10^{-304}$)~\cite{anscombe1983distribution}.
     An artefact in the data is that the mean of all of the points for the coastline
     seems to be uniformly shifted upwards between the two years (A~\&~E),
     where the difference is $0.06\pm0.01$m. }
   \label{fig:ssh_stats_america}
\end{figure*}

Initially looking at the statistics of the sea surface height along the
US coast in Figure~\ref{fig:ssh_stats_america}.

The sea surface moves up and down.

The ocean is known to have a red-noise spectrum.

The relationship between the temperature and the sea surface temperature.

The Florida current is a likely source of noise at Miami.

\begin{equation}
F(f)=\int_{-\infty}^{\infty} y(x) e^{-i 2\pi f x} d x
\end{equation}
\begin{equation}
\sum_{n=0}^{N-1} a_{n} e^{-2 \pi i  k n/ N}=\sum_{n=0}^{N / 2-1} a_{2 n}
e^{-2 \pi i k (2 n)/ N} +\sum_{n=0}^{N / 2-1} a_{2n+1} e^{-2 \pi i k (2 n+1)/ N}
\end{equation}

\begin{figure*}
\includegraphics[width=0.48\linewidth]{../surge/plots/low_pass_grid.pdf}
     \includegraphics[width=0.48\linewidth]{../surge/plots/high_pass_grid.pdf}
\caption{Fourier Transform~\cite{cooley1965algorithm} of SSH demeaned by
         coastal point, and low (left) and high (right) pass filtered respectively,
         with a threshold of 2 yr$^{-1}$ (justified by
         Figure~\ref{fig:lpredthresh}). Although not noticeably different in
         the low pass filter (left), Miami and other headlands
         are noticeably less variable in the high pass filter. }
\label{fig:four-trans-panels}
\centering
\includegraphics[width=0.8\linewidth]{../surge/plots/norlean-low-pass.pdf}
\vspace{-10pt}

\caption{New Orleans (NO) closest coastal cell divided
         into high and low frequency components.}
\label{fig:no-hp-lp}


\includegraphics[width=0.8\linewidth]{../surge/plots/mm-low-pass.pdf}
\vspace{-10pt}

\caption{Miami (MM) has more low frequency noise.
         Both this and Figure~\ref{fig:no-hp-lp} above
         show an increase in the low amplitude signal with time,
         which could be explained by the spin
         up of \texttt{tyr} from EN4.}
\label{fig:mm-hp-lp}

\end{figure*}



\begin{figure*}
\centering

\vspace{-35pt}
       \includegraphics[width=0.27\linewidth]{../surge/plots/reg_fft/up_to_100_full_coast.png}
        \includegraphics[width=0.35\linewidth]{../surge/plots/reg_fft/up_to_100_coastal_average.png}
            \caption{\texttt{eUS-tyr} An averaged $r^2$ between \texttt{huber} and \texttt{MLR} trained on 2004 and 2005,
                    and tested on the opposite year (see §~\ref{sec:responsiveness}).}
            \label{fig:lpredthresh}

\begin{minipage}{0.45\textwidth}
\includegraphics[width=1\linewidth]{../surge/plots/lag/fft_pp_zos.png}
            \includegraphics[width=\linewidth]{../surge/plots/lag/zm_fft_pp_zos.png}
            \caption{\texttt{eUS} for \texttt{tyr} demeaned SSH, $\Delta\eta$, fourier transform. Bottom plot is a zoomed in version of the upper.}
            \label{fig:zm_fft_zos}
            \end{minipage}\begin{minipage}{0.45\textwidth}

        \includegraphics[width=1\linewidth]{../surge/plots/lag/fft_pp_tos.png}
            \includegraphics[width=\linewidth]{../surge/plots/lag/zm_fft_pp_tos.png}
            \caption{\texttt{eUS} for \texttt{tyr} demeaned SST, $\Delta T_s$, fourier transform,
                     showing power c.~uniquely located in annual signal.}
            \label{fig:zm_fft_tos}
            \end{minipage}

\end{figure*}


\subsection{Lag Correlation Plot}
\label{sec:lag}

There is a lag, between the two, which becomes anti-phase in the Gulf
of Tonkin (Northern Vietnam).

Some level of lag is understandable, as the mixed layer takes longer
to heat up than the surface itself.

\begin{figure*}
\includegraphics[width=1\linewidth]{../surge/plots/temp/ssh_sst_grid.pdf}
\caption{\texttt{tyr}, \texttt{eUS} SSH and SST.
         There is some interesting structure downstream of Miami, could this be
         movements in the Florida current?
         If so, could it be predictive of the SSH upstream}
         \label{fig:ssh-sst}

\centering
\begin{minipage}{0.45\textwidth}
\includegraphics[width=1\linewidth]{../surge/plots/lag/correlate.pdf}
         \caption{Lag correlation plot for \texttt{eUS} for \texttt{tyr}.
                  The SSH is a different amount behind the
                  SST.}
         \label{fig:lag-plot}
\end{minipage}
\begin{minipage}{0.45\textwidth}
\includegraphics[width=1\linewidth]{../surge/plots/lag/vc_correlate.pdf}
         \caption{Lag correlation plot for \texttt{vc} for \texttt{tyr}.
         Becomes anti-phase in N~Vietnam / S~China, and perfectly
         in phase in N~China.}
         \label{fig:lag-plot}
\end{minipage}
\end{figure*}

