\documentclass[usenames, dvipsnames]{beamer}
% Copyright 2019 Clara Eleonore Pavillet, Adapted 2020 by Simon Thomas (for Cambridge Colours etc.)

% Authors: Clara Eleonore Pavillet & Simon Donald Alistair Thomas
% Description: This is an unofficial Cambridge University Beamer Template I made from scratch.
%  Feel free to use it, modify it, share it.
% Version: 2.0\usepackage[utf8]{inputenc}
\usepackage[dvipsnames]{xcolor}
\usepackage{tikz}
\usetikzlibrary{positioning, calc}
\usepackage{graphicx}
\usepackage{hyperref}
\usepackage{amsmath}
\usepackage{listings}
\usepackage{fontawesome}
\usepackage{appendixnumberbeamer}
\usepackage{Theme/beamercolorthemeoxonian}
\usepackage{Theme/beamerfontthemeoxonian}
\usepackage{Theme/beamerinnerthemeoxonian}
\usepackage{Theme/beamerouterthemeoxonian}
\usepackage{Theme/beamerthemeoxonian}
\RequirePackage{amsmath}
\RequirePackage{amssymb}
% Define Commands
\newcommand*{\ClipSep}{0.06cm} %To adjust footer logo
\newcommand{\E}{\mathrm{e}\,} %\def\I{e} % used to defined e for exp(x), see later what it should be
\newcommand{\ud}{\mathrm{d}}
\lstset{numbers=left,
        numberstyle=\tiny,
        stepnumber=1,
        firstnumber=1,
        breaklines=true,
        numbersep=5pt,
        language=Python,
        stringstyle=\ttfamily,
        basicstyle=\footnotesize,
        showstringspaces=false
}
% Define Colours
\definecolor{cambridgeblue}{RGB}{163, 193, 173}
\definecolor{oxfordblue}{RGB}{0, 32, 68}
\definecolor{glassgrey}{rgb}{220, 220, 220}

\RequirePackage[
  bibstyle=nature,
  citestyle=numeric,
  isbn=true,
  doi=true,
  sorting=none,
  url=true,
  urldate = long,
  % defernumbers=true,
  bibencoding=utf8,
  backend=bibtex
]{biblatex}
\RequirePackage{xpatch}
\RequirePackage{url}
\RequirePackage{hyperref}
\hypersetup{
    colorlinks=true,
    linkcolor=oxfordblue,
    filecolor=BrickRed,
    citecolor=BrickRed,
    urlcolor=BrickRed,
}

\graphicspath{{images/}{../images/}}

\addbibresource{references/surge.bib}
\addbibresource{references/generic_references.bib}
\addbibresource{references/references.bib}
\addbibresource{references/fluid_dynamics.bib}
\addbibresource{references/machine_learning.bib}
\addbibresource{references/global_warming.bib}
\addbibresource{references/programming.bib}
\addbibresource{references/taleb.bib}
\addbibresource{references/tebutt.bib}

\addbibresource{references/generic_references.bib}
\addbibresource{references/references.bib}
\addbibresource{references/fluid_dynamics.bib}
\addbibresource{references/machine_learning.bib}
\addbibresource{references/global_warming.bib}
\addbibresource{references/programming.bib}
\addbibresource{references/surge.bib}
\addbibresource{references/tebbutt.bib}
\addbibresource{references/taleb.bib}

\title{Using Learning Algorithms to Investigate East~US Storm Surges \vspace{-15pt}}
\titlegraphic{\includegraphics[height=1.25cm]{Theme/Logos/UCC.png}\hspace{0.5cm}\includegraphics[height=1.25cm]{Theme/Logos/small-bas.png}}
\author{Simon D.\ A.\ Thomas}
\institute{University of Cambridge | British Antarctic Survey\\
\href{mailto:sdat2@cam.ac.uk}{sdat2@cam.ac.uk} | \href{mailto:sithom@bas.ac.uk}{sithom@bas.ac.uk} }
\date{Collaborating with Dr.s:\\ Dan Jones, Laure Zanna, Pierre Mathiot, \& Rory Bingham } %\today
\immediate\write18{texcount -nc -inc -merge -sum -q proposal.tex > wordcount/pres_wordcount.tex}
\begin{document} %%%%%%%%%%%%%%%%%%%%%%%%%%%%%%%%%%%%%%%%%%%%%%%%%%%%%%%%%%%%%%%
{\setbeamertemplate{footline}{}
\frame{\titlepage}}
%%%%%%%%%%%%%%%%%%%%%%%%%%%%%%%%%%%%%%%%%%%%%%%%%%%%%%%%%%%%%%%%%%%%%%%%%%%%%%%%
\subsection{Welcome}
\begin{frame}{Motivation}
\vspace{-10pt}
We expect climate change to increase storm surge hazard~\cite{SROCC} as:\\
\begin{itemize}
\item Hotter surface temperatures increase the intensity of tropical cyclones (TCs),
      as their ability to do work comes from the heat contrast between their
      bottom and top~\cite{emanuel1991theory}.
 \item Sea level rise puts more areas at risk (see e.g.~\cite{kulp2019new}).
 \end{itemize}
\textbf{But by how much, where, and what is the error?}

\begin{figure}[htb!]
    \includegraphics[width=0.9\linewidth]{images/example-images/new-orleans.jpg}
\end{figure}
\vspace{5pt}
\end{frame}

\begin{frame}{Cambridge Flooding Prediction~\cite{kulp2019new, kulp2018coastaldem}}
\vspace{-30pt}
\begin{figure}[htb!]
    \centering
    \hspace{-20pt}
    \includegraphics[width=0.9\paperwidth]{images/example-images/cambridge-surge.png}
    \vspace{-7pt}
    \caption{\url{https://coastal.climatecentral.org/map/}}
    \label{fig:}
\end{figure}
\end{frame}

\begin{frame}{New Orleans Flooding Prediction~\cite{kulp2019new, kulp2018coastaldem}}
\vspace{-20pt}
\begin{figure}[htb!]
    \centering
   \hspace{-20pt}
    \includegraphics[width=0.9\paperwidth]{images/example-images/new-orleans-surge.png}
    \vspace{-7pt}
    \caption{\url{https://coastal.climatecentral.org/map/}}
    \label{fig:}
\end{figure}
\end{frame}

\begin{frame}{Structure}

This talk covers:
\vspace{5pt}

\begin{enumerate}
    \item A brief introduction to ORCA12 (the model used).
    \item The method of extraction of data from the model.
    \item Computing and justifying local coastal metrics.
     \item Initial local regression results.
\end{enumerate}

\vspace{5pt}
This work was inspired by Yin et al.~2020~\cite{ZannaPreprint}.\\
Please contact me if you have any suggestions!
\href{mailto:sdat2@cam.ac.uk}{sdat2@cam.ac.uk}
or \href{mailto:sithom@bas.ac.uk}{sithom@bas.ac.uk}.
\begin{center}
\end{center}
\end{frame}


\begin{frame}{The NEMO ORCA12 Ocean Model~\cite{madec2015nemo}}
\vspace{-20pt}
\begin{figure}[htb!]
    \centering
    \hspace{-30pt}\includegraphics[width=0.74\linewidth]{images/example-images/fig-irregular.png}
     \includegraphics[width=0.30\linewidth]{images/example-images/nemo-poles.png}

    \vspace{-7pt}

    \caption{NEMO ORCA12 uses a tripolar ocean grid so that no coordinate singularities are in the ocean.
     The resolution is $\frac{1}{12}^{\circ}$.
      Initial condition EN4~\cite{good2013en4, HadObs} (start of the run = 1976).
Forcing CORE2~\cite{griffies2012datasets,large2009global}.}

    \label{fig:}
\end{figure}
\end{frame}


\begin{frame}{The HADSGEM3 CMIP6 Model~\cite{williams2018met, FurtherInfo}}
\vspace{-20pt}
\begin{itemize}
\item `\texttt{control-1950}' to 2050. Daily mean outputs~\cite{williams2018met, FurtherInfo}.
\item This experiment is not standard to CMIP6~\cite{eyring2016overview}.
\item There is a problem with representing TCs~\cite{tomassini2017interaction}:

\end{itemize}
\begin{figure}
        \includegraphics[width=0.7\linewidth]{images/HAD-TC.png}
            \caption{The models all have
            significantly less TCs in the North Atlantic
             than observations.
             \textit{Figure 19 from~\cite{williams2018met}.} }

\end{figure}
\end{frame}

\begin{frame}{Systematic Error: Wind Stress Parameterisation}
\vspace{-40pt}
\centering
\begin{equation}
 |\tau| = c_d(|U|) \cdot U^2
 \end{equation}

    \hspace{-30pt}\begin{minipage}{0.45\textwidth}
    \begin{figure}
            \includegraphics[width=1\linewidth]{images/example-images/cd.pdf}
                \caption{Fig. 3c from Powell 2003~\cite{powell2003reduced},
                 where they suggested that at high wind speeds $c_d$ decreases.}

    \end{figure}

    \end{minipage}\hspace{5pt}
      \begin{minipage}{0.57\textwidth}
\begin{figure}[htb!]
    \centering
    \includegraphics[width=1\linewidth]{../surge/plots/cd_finder.pdf}
    \caption{ Extracting $C_d$ from data.
    $ r_p = 0.9774 \pm 0.001,\;\; p<10^{-300}$\\
    $ m = 0.0383 \pm 0.0001 $ kg$^{0.5}$ m$^{-1.5}$\\
    $\implies  c_d = 0.001467 \pm 0.000008$ kg m$^{-3}$
    % error estimated from differences in output between test/training year,
    % measured regression error in either year much smaller than this.
    % The error is propogated using the \texttt{python3.uncertanties} package}
    }
    \label{fig:}
\end{figure}
    \end{minipage}
\end{frame}


\begin{frame}{Helpful Place Labels}
\vspace{-20pt}
\begin{figure}[htb!]
    \centering
    \includegraphics[width=0.85\linewidth]{../surge/plots/point_choosing/standard_loc_test.pdf}
    \vspace{-7pt}
    \caption{Places that I will label the x-axis with.
     The coastline is a self-similar fractal~\cite{mandelbrot1967long, richardson1961problem},
      hence coast length is dependent on resolution.
    [Fractal dimension changes along the coast.~\cite{jiang1998fractal}]}
    \label{fig:}
\end{figure}
\end{frame}


\begin{frame}{Automatically Point Selection from ORCA12}
\vspace{-20pt}
\begin{figure}[htb!]
    \centering
    \hspace{-10pt}
    \includegraphics[width=0.60\linewidth]{../surge/plots/quiver_plot.pdf}
    \hspace{5pt}
    \includegraphics[width=0.38\linewidth]{images/example-images/new-orleans-example.png}
    \vspace{-7pt}
    \caption{The full selection of coastal points taken from ORCA12. Green arrows show
             orientation of that cell's stretch of coast line.}
    % \label{fig:}
\end{figure}
\end{frame}

\section{Local Coastal Metrics}
    \begin{frame}[plain]
        \vfill
      \centering
      \begin{beamercolorbox}[sep=8pt,center,shadow=true,rounded=true]{title}
        \usebeamerfont{title}\insertsectionhead\par%
        \color{oxfordblue}\noindent\rule{10cm}{1pt} \\
        \includegraphics[width=0.4\linewidth]{images/example-images/new-orleans-example.png}
        \includegraphics[width=0.52\linewidth]{../surge/plots/angles_hist.pdf}
      \end{beamercolorbox}
      \vfill
  \end{frame}


\begin{frame}{The change in angles along the coastline}
\vspace{-20pt}
\begin{figure}[htb!]
    \centering
    \begin{equation}
B_i^{\prime}=\operatorname{arctan} 2\left(\sum_{j=i-4\sigma}^{j=i+4\sigma}
     \mathcal{N}(i, \sigma^{2})\cdot \sin{( B_{j})},\; \sum_{j=i-4\sigma}^{j=i+4\sigma}
      \mathcal{N}(i, \sigma^{2})\cdot  \cos{( B_{j})}\right)
\end{equation}
    \includegraphics[width=1.0\linewidth]{../surge/plots/angle_heatmap.pdf}
    % \caption{To deal with this we could smooth, and try a variety of $\sigma$.}
    % \label{fig:}
\end{figure}
\end{frame}


\begin{frame}{Angular derivative signal noisy for small $\sigma$}
\vspace{-20pt}
\begin{figure}[htb!]
    \centering
        \begin{equation}
        \frac{\partial B_i^{\prime}}{\partial \mathrm{pt}} \equiv \frac{B_{i+1}^{\prime}-B_{i-1}^{\prime}}{2}
\end{equation}
    \includegraphics[width=1.0\linewidth]{../surge/plots/derivative_heatmap_1_11.pdf}
    % \caption{To deal with this we could smooth, and try a variety of $\sigma$.}
    % \label{fig:}
\end{figure}
\end{frame}

%    plot_derivative_angle_heatmap(1, 11)
%    plot_derivative_angle_heatmap(10, 30)
%    plot_derivative_angle_heatmap(30, 100)

\begin{frame}{Better for  moderate $\sigma$}
\vspace{-20pt}
\begin{figure}[htb!]
    \centering
            \begin{equation}
        \frac{\partial B_i^{\prime}}{\partial \mathrm{pt}}
        \equiv \frac{B_{i+1}^{\prime}-B_{i-1}^{\prime}}{2} \notag
\end{equation}
    \includegraphics[width=1.0\linewidth]{../surge/plots/derivative_heatmap_10_30.pdf}
    % \caption{To deal with this we could smooth, and try a variety of $\sigma$.}
    % \label{fig:}
\end{figure}
\end{frame}

\begin{frame}{Unresponsive for high $\sigma$}
\vspace{-20pt}
\begin{figure}[htb!]
    \centering
                \begin{equation}
        \frac{\partial B_i^{\prime}}{\partial \mathrm{pt}}
        \equiv \frac{B_{i+1}^{\prime}-B_{i-1}^{\prime}}{2} \notag
\end{equation}
    \includegraphics[width=1.0\linewidth]{../surge/plots/derivative_heatmap_30_100.pdf}
    % \caption{To deal with this we could smooth, and try a variety of $\sigma$.}
    % \label{fig:}
\end{figure}
\end{frame}


\begin{frame}{Extracting Bathymetry}
\vspace{-30pt}
\hspace{-30pt}
\begin{figure}[htb!]
    \centering
    \hspace{-30pt}\includegraphics[width=1.1\linewidth]{../surge/plots/bath_list.pdf}
    % \label{fig:}
\end{figure}
\end{frame}


\begin{frame}{Extracting Bathymetry}
\vspace{-30pt}
\hspace{-30pt}
\begin{figure}[htb!]
    \centering
    \hspace{-35pt}\includegraphics[width=1.1\linewidth]{../surge/plots/distance_isobath.pdf}
    % \label{fig:}
\end{figure}
\end{frame}

\begin{frame}{Isobath Distances are Heavily Correlated}
\vspace{-30pt}
\hspace{-30pt}
\begin{figure}[htb!]
    \centering
    \hspace{-35pt}\includegraphics[width=0.65\linewidth]{../surge/plots/isobath_correlate.pdf}
\end{figure}
\end{frame}

\begin{frame}{An Example around New Orleans}
\vspace{-30pt}
\begin{figure}[htb!]
    \centering
    \includegraphics[width=0.75\linewidth]{../surge/plots/new_orleans_map.pdf}
\end{figure}
\end{frame}

\begin{frame}{A Katrina-Like Hurricane}
\vspace{-30pt}
\begin{figure}[htb!]
    \centering
    \includegraphics[width=0.75\linewidth]{../surge/plots/katrina_graph.pdf}
\end{figure}
\end{frame}

\section{Sea Surface Velocity along Y-axis, (Vos, $v$) }
    \begin{frame}[plain]
        \vfill
      \centering
      \begin{beamercolorbox}[sep=8pt,center,shadow=true,rounded=true]{title}
        \usebeamerfont{title}\insertsectionhead\par%
        \color{oxfordblue}\noindent\rule{10cm}{1pt} \\
        \includegraphics[width=0.93\linewidth]{images/example-images/vos.png}
      \end{beamercolorbox}
      \vfill
  \end{frame}

\section{Sea Surface Height above Reference Geoid (Zos, $\eta$, SSH) }
    \begin{frame}[plain]
        \vfill
      \centering
      \begin{beamercolorbox}[sep=8pt,center,shadow=true,rounded=true]{title}
        \usebeamerfont{title}\insertsectionhead\par%
        \color{oxfordblue}\noindent\rule{10cm}{1pt} \\
        \includegraphics[width=0.93\linewidth]{images/example-images/zos-image.png}
      \end{beamercolorbox}
      \vfill
  \end{frame}

  \begin{frame}{$\eta$ Covariance and Correlation matrices are similar. }
\vspace{-20pt}
\begin{figure}[htb!]
    \centering
    \hspace{-10pt}
    \includegraphics[width=0.68\linewidth]{../surge/plots/corr_cov.pdf}
            \includegraphics[width=0.16\linewidth]{../surge/plots/corr_cov_cbar.pdf}
    \vspace{-7pt}
    \caption{A/B: Covariance matrices for 2004/5.\\
             $\quad\quad\quad\;\;$C/D: Correlation matrices for 2004/5.}
    \label{fig:}
\end{figure}
\end{frame}

\begin{frame}{ $\eta$ (SSH/zos) similar between 2004/5 }
\vspace{-20pt}
\begin{figure}[htb!]
    \centering
    \includegraphics[width=0.6\linewidth]{../surge/plots/stats_points_plot.pdf}
       \hspace{0pt} \includegraphics[width=0.285\linewidth]{../surge/plots/stats_points_plot_1.pdf}
    \vspace{-7pt}
    \caption{A comparison between the distributions of sea surface height
     above geoid (zos) for the training set (2005) and the test set (2004).}
   % \label{fig:}
\end{figure}
\end{frame}

\begin{frame}{$\eta$ No clear discontinuity between 2004/5  }
\vspace{-20pt}
\begin{figure}[htb!]
    \centering
    \includegraphics[width=0.95\linewidth]{../surge/plots/stats_points_plot_2.pdf}
    \vspace{-7pt}
    \caption{An attempt to check whether there are discontinuities
     over the special points, to explain the offsets of the means.}
    %\label{fig:}
\end{figure}
\end{frame}

\begin{frame}{Spin up may explain displacement in $\eta$ . }
\vspace{-20pt}
\begin{figure}[htb!]
    \centering
    \includegraphics[width=0.95\linewidth]{../surge/plots/stats_points_plot_5.pdf}
    \vspace{-7pt}
    \caption{Early months in 2004 do not seem to follow the yearly pattern followed elsewhere.}
    %\label{fig:A}
\end{figure}
\end{frame}


\begin{frame}{Gaussian Processes Refresher}
\vspace{-20pt}
\begin{itemize}
\item Equations 2.13-2.14 in GPML~\cite{williams2006gaussian}.
\begin{align}
m(\mathbf{x})&=&\mathbb{E}[f(\mathbf{x})] % \tag{Mean}
\\
k\left(\mathbf{x}, \mathbf{x}^{\prime}\right)&=&\mathbb{E}
\left[(f(\mathbf{x})-m(\mathbf{x}))\left(f\left(\mathbf{x}^{\prime}\right)
-m\left(\mathbf{x}^{\prime}\right)\right)\right]
%\tag{Covariance}
\\
f(\mathbf{x})& \sim& \mathcal{G} \mathcal{P}\left(m(\mathbf{x}),
 k\left(\mathbf{x}, \mathbf{x}^{\prime}\right)\right)%\tag{GP}
\end{align}
\item Can assume $m(\mathbf{x})=0$ without terrible consequences.
 \item Thought should be put into the form of $k\left(\mathbf{x},
  \mathbf{x}^{\prime}\right)$~\cite{duvenaud2014automatic}.
\end{itemize}
\end{frame}

\begin{frame}{Warped Gaussian Processes}
\begin{itemize}
\item GPs assume that there is a Gaussian error around each point, but this is often not the case in real variables.
However, it is often possible to transform to a space where this is the case, Krige there,
 and then transform back~\cite{snelson2004warped}.\footnote{\url{http://mlg.eng.cam.ac.uk/zoubin/papers/gpwarp.pdf}}
 \item Lewis Fry Richardson (1948) showed that the estimates of deaths during a war had
 symmetric error bars in logarithmic space~\cite{richardson1948variation}.
 \end{itemize}
\end{frame}


\begin{frame}{There is a strong yearly periodicity in $\eta$. }
\vspace{-20pt}
\begin{figure}[htb!]
    \centering
    \includegraphics[width=0.95\linewidth]{../surge/plots/ahh/ahhhhh4.pdf}
    \vspace{-7pt}
    \caption{There is a strong yearly periodicity in $\eta$ (and its variance?).
     Kriged with a Sobol quasirandom subsample~\cite{sobol1967distribution} of 6000 points.}
    %\label{fig:A}
\end{figure}
\end{frame}


\begin{frame}{There is  supported by taking the Fourier Transform }
\vspace{-40pt}
\begin{figure}[htb!]
    \centering
    \begin{equation}
F(f)=\int_{-\infty}^{\infty} y(x) e^{-i 2\pi f x} d x
\end{equation}
\begin{equation}
\sum_{n=0}^{N-1} a_{n} e^{-2 \pi i  k n/ N}=\sum_{n=0}^{N / 2-1} a_{2 n}
e^{-2 \pi i k (2 n)/ N} +\sum_{n=0}^{N / 2-1} a_{2n+1} e^{-2 \pi i k (2 n+1)/ N}
\end{equation}
    \includegraphics[width=0.85\linewidth]{../surge/plots/fourier_transform_grid.pdf}
    \vspace{-7pt}
    \caption{Fourier Transform~\cite{cooley1965algorithm} of $\Delta\eta$ by coastal point.}
    % \label{fig:A}
\end{figure}
\end{frame}


\begin{frame}[fragile]{Low / High Pass Filter of New Orleans }
\vspace{-10pt}
    %\centering
\begin{block}{Fast Fourier Transform (1) and Its Inverse (2)}
\vspace{-10pt}
\begin{lstlisting}[firstnumber=1, language=python, label=glabels, xleftmargin=0pt]
y(j) = (x * exp(-2*pi*sqrt(-1)*j*np.arange(n)/n)).sum()
y(j) = (x * exp(2*pi*sqrt(-1)*j*np.arange(n)/n)).mean()
\end{lstlisting}
\end{block}
\vspace{-30pt}
\begin{figure}[htb!]
\begin{equation}
\Delta\eta_{\;\mathrm{lp}}(t) = \int_{\mathbb{R}}W\left(\frac{|f|}
{|f|_{\mathrm{thresh}}}\right)\int_{\mathbb{R}}e^{2\pi i
(t-t^{\prime})f }\Delta \eta(t^{\prime})dt^{\prime}df
\end{equation}
\begin{equation}
\Delta\eta_{\;\mathrm{hp}}(t) = \int_{\mathbb{R}}\left(1-W\left(\frac{|f|}
{|f|_{\mathrm{thresh}}}\right)\right)\int_{\mathbb{R}}e^{2\pi i
(t-t^{\prime})f }\Delta \eta(t^{\prime})dt^{\prime}df
\end{equation}
    \centering
    \includegraphics[width=0.95\linewidth]{../surge/plots/norlean-low-pass.pdf}
    \vspace{-15pt}
    \caption{Fourier Transform~\cite{cooley1965algorithm} of $\Delta\eta$ for New Orleans.}
    \label{fig:A}
\end{figure}
\end{frame}

\begin{frame}[fragile]{Low / High Pass Filter of Miami }
\vspace{-10pt}
    %\centering
\begin{block}{Fast Fourier Transform (1) and Its Inverse (2)}
\vspace{-10pt}
\begin{lstlisting}[firstnumber=1, language=python, label=glabels, xleftmargin=0pt]
y(j) = (x * exp(-2*pi*sqrt(-1)*j*np.arange(n)/n)).sum()
y(j) = (x * exp(2*pi*sqrt(-1)*j*np.arange(n)/n)).mean()
\end{lstlisting}
\end{block}
\vspace{-30pt}
\begin{figure}[htb!]
\begin{equation}
\Delta\eta_{\;\mathrm{lp}}(t) = \int_{\mathbb{R}}W\left(\frac{|f|}
{|f|_{\mathrm{thresh}}}\right)\int_{\mathbb{R}}e^{2\pi i (t-t^{\prime})f }
\Delta \eta(t^{\prime})dt^{\prime}df
\end{equation}
\begin{equation}
\Delta\eta_{\;\mathrm{hp}}(t) = \int_{\mathbb{R}}\left(1-W\left(\frac{|f|}
{|f|_{\mathrm{thresh}}}\right)\right)\int_{\mathbb{R}}e^{2\pi i (t-t^{\prime})f}
   \Delta \eta(t^{\prime})dt^{\prime}df
\end{equation}
    \centering
    \includegraphics[width=0.95\linewidth]{../surge/plots/mm-low-pass.pdf}
    \vspace{-15pt}
    \caption{Fourier Transform~\cite{cooley1965algorithm} of $\Delta\eta$ for Miami.}
    \label{fig:A}
\end{figure}
\end{frame}

\begin{frame}[fragile]{Low / High Pass Filter of New York }
\vspace{-30pt}
    %\centering
\begin{block}{Fast Fourier Transform (1) and Its Inverse (2)}
\begin{lstlisting}[firstnumber=1, language=python, label=glabels, xleftmargin=0pt]
y(j) = (x * exp(-2*pi*sqrt(-1)*j*np.arange(n)/n)).sum()
y(j) = (x * exp(2*pi*sqrt(-1)*j*np.arange(n)/n)).mean()
\end{lstlisting}
\end{block}
\vspace{-20pt}
\begin{figure}[htb!]
    \centering
    \includegraphics[width=0.95\linewidth]{../surge/plots/ny-low-pass.pdf}
    \vspace{-7pt}
    \caption{Fourier Transform~\cite{cooley1965algorithm} of $\Delta\eta$ for New York.}
    \label{fig:A}
\end{figure}
\end{frame}


\begin{frame}{Low / High Pass Filter for all Points}
\vspace{-30pt}
    %\centering
\hspace{-30pt}
 \begin{minipage}{1.15\textwidth}
\begin{figure}[htb!]
    \centering
   \hspace{-40pt} \includegraphics[width=0.48\linewidth]{../surge/plots/low_pass_grid.pdf}
        \includegraphics[width=0.48\linewidth]{../surge/plots/high_pass_grid.pdf}
    \vspace{-7pt}
    \caption{Fourier Transform~\cite{cooley1965algorithm} of $\Delta\eta$}
    \label{fig:A}
\end{figure}
\end{minipage}
\end{frame}


\begin{frame}{$|f|_{thresh}=2$ yr$^{-1}$ Maximises Linear Predictability }
\centering

Linear Predictability (LP) defined in next section

    \hspace{-30pt}\begin{minipage}{0.57\textwidth}
    \begin{figure}
            \includegraphics[width=1\linewidth]{../surge/plots/reg_fft/up_to_100_full_coast.pdf}
                \caption{LP for each point.}
    \end{figure}

    \end{minipage}\hspace{5pt}
      \begin{minipage}{0.45\textwidth}
\begin{figure}[htb!]
    \centering
    \includegraphics[width=1\linewidth]{../surge/plots/reg_fft/up_to_100_coastal_average.pdf}
    \caption{Averaged over points.}
    \label{fig:}
\end{figure}
    \end{minipage}
\end{frame}


\section{Tau-Tau Plots}
    \begin{frame}[plain]
        \vfill
      \centering
      \begin{beamercolorbox}[sep=8pt,center,shadow=true,rounded=true]{title}
        \usebeamerfont{title}\insertsectionhead\par%
        \color{oxfordblue}\noindent\rule{10cm}{1pt} \\
        \begin{itemize}
        \item As we already noted $\tau$ goes approximately quadratically with $U$.
        \item Decomposed into $\tau_u$, $\tau_v$ (along u, v axes).
        \item Regressing $\Delta\eta_{\quad hp}$ against $\tau_u$ and $\tau_v$ seems to
         reveal a linear relationship, most clear at vulnerable sights (New Orleans).
        \item How much of a difference does Robust vs. Standard Linear Regression make?
        \end{itemize}
       % \includegraphics[width=0.4\linewidth]{images/example-images/new-orleans-example.png}
        %\includegraphics[width=0.52\linewidth]{../surge/plots/angles_hist.pdf}
      \end{beamercolorbox}
      \vfill
  \end{frame}

\begin{frame}{New Orleans $\Delta\eta$ or $\Delta\eta_{\;\;\mathrm{hp}}$ against $\tau_u$, or $\tau_v$}
\vspace{-30pt}
    %\centering
\hspace{-30pt}
 \begin{minipage}{1.15\textwidth}
\begin{figure}[htb!]
    \centering
   \hspace{-40pt} \includegraphics[width=0.42\linewidth]{../surge/plots/tau-tau/norlean-u.pdf}
        \includegraphics[width=0.42\linewidth]{../surge/plots/tau-tau/norlean-v.pdf}

   \hspace{-40pt} \includegraphics[width=0.42\linewidth]{../surge/plots/tau-tau-hp/norlean-u.pdf}
        \includegraphics[width=0.42\linewidth]{../surge/plots/tau-tau-hp/norlean-v.pdf}
    \vspace{-15pt}
    \caption{High pass filter changes New Orleans little.}
    \label{fig:A}
\end{figure}
\end{minipage}
\end{frame}

\begin{frame}{Miami $\Delta\eta$ or $\Delta\eta_{\;\;\mathrm{hp}}$ against $\tau_u$, or $\tau_v$}
\vspace{-30pt}
    %\centering
\hspace{-30pt}
 \begin{minipage}{1.1\textwidth}
\begin{figure}[htb!]
    \centering
   \hspace{-40pt} \includegraphics[width=0.42\linewidth]{../surge/plots/tau-tau/miami-u.pdf}
        \includegraphics[width=0.42\linewidth]{../surge/plots/tau-tau/miami-v.pdf}

   \hspace{-40pt} \includegraphics[width=0.42\linewidth]{../surge/plots/tau-tau-hp/miami-u.pdf}
        \includegraphics[width=0.42\linewidth]{../surge/plots/tau-tau-hp/miami-v.pdf}
    \vspace{-15pt}
    \caption{High pass filter makes Miami more New Orleans like.}
    \label{fig:A}
\end{figure}
\end{minipage}
\end{frame}


\begin{frame}{New York $\Delta\eta$ or
              $\Delta\eta_{\;\;\mathrm{hp}}$ against $\tau_u$, or $\tau_v$}
\vspace{-30pt}
    %\centering
\hspace{-30pt}
 \begin{minipage}{1.1\textwidth}
\begin{figure}[htb!]
    \centering
   \hspace{-40pt} \includegraphics[width=0.42\linewidth]{../surge/plots/tau-tau/new-york-u.pdf}
        \includegraphics[width=0.42\linewidth]{../surge/plots/tau-tau/new-york-v.pdf}

   \hspace{-40pt} \includegraphics[width=0.42\linewidth]{../surge/plots/tau-tau-hp/new-york-u.pdf}
        \includegraphics[width=0.42\linewidth]{../surge/plots/tau-tau-hp/new-york-v.pdf}
    \vspace{-15pt}
    \caption{High pass filter (lower two panels) makes small difference.}
    \label{fig:A}
\end{figure}
\end{minipage}
\end{frame}

\section{But these plots all had high excess kurtosis on all variables!  }
    \begin{frame}[plain]
        \vfill
      \centering
      \begin{beamercolorbox}[sep=8pt,center,shadow=true,rounded=true]{title}
        \usebeamerfont{title}\insertsectionhead\par%
        \color{oxfordblue}\noindent\rule{10cm}{1pt} \\
        \begin{itemize}
        \item$\mathbb{P}(\mathrm{Gaussian})<10^{-304}$ etc.
        \item The variables are also not necessarily positive (so no $\log$ trick).
        \item It would be interesting to see if there is a warping function
         that can do this (Extreme Value Textbooks are being rapidly skimmed).
        \item E.g .       \begin{equation}
       x^{\prime} = \tanh{\frac{x}{2\sigma_x}} \text{}
       \end{equation}
       Does make the distributions more normal.
        \end{itemize}
       % \includegraphics[width=0.4\linewidth]{images/example-images/new-orleans-example.png}
        %\includegraphics[width=0.52\linewidth]{../surge/plots/angles_hist.pdf}
      \end{beamercolorbox}
      \vfill
  \end{frame}


\begin{frame}{New York $\Delta\eta_{\;\;\mathrm{hp}}$ against $\tau_u$, and $\tau_v$}
\vspace{-30pt}
    %\centering
\hspace{-30pt}
 \begin{minipage}{1.1\textwidth}

\begin{figure}[htb!]
    \centering
   \hspace{-40pt} \includegraphics[width=1\linewidth]{../surge/plots/3d_plots/3d_plotnew-york.pdf}
    \vspace{-15pt}
   % \caption{High pass filter (lower two panels) makes small difference.}
    \label{fig:A}
\end{figure}
\end{minipage}
\end{frame}


\begin{frame}{Miami $\Delta\eta_{\;\;\mathrm{hp}}$ against $\tau_u$, and $\tau_v$}
\vspace{-30pt}
    %\centering
\hspace{-30pt}
 \begin{minipage}{1.1\textwidth}

\begin{figure}[htb!]
    \centering
   \hspace{-40pt} \includegraphics[width=1\linewidth]{../surge/plots/3d_plots/3d_plotmiami.pdf}
    \vspace{-15pt}
    \caption{Most variance not attributable to wind stress.}
    \label{fig:A}
\end{figure}
\end{minipage}
\end{frame}


\begin{frame}{New Orleans $\Delta\eta_{\;\;\mathrm{hp}}$ against $\tau_u$, and $\tau_v$}
\vspace{-30pt}
    %\centering
\hspace{-30pt}
 \begin{minipage}{1.1\textwidth}
\begin{figure}[htb!]
    \centering
   \hspace{-40pt} \includegraphics[width=1\linewidth]{../surge/plots/3d_plots/3d_plotnorlean.pdf}
    \vspace{-15pt}
   % \caption{High pass filter (lower two panels) makes small difference.}
    \label{fig:A}
\end{figure}
\end{minipage}
\end{frame}


\begin{frame}{Regression Summarised for Every Point.}
\vspace{-30pt}
    %\centering
\hspace{-30pt}
 \begin{minipage}{1.1\textwidth}
\begin{figure}[htb!]
    \centering
   \hspace{-40pt} \includegraphics[width=0.7\linewidth]{../surge/plots/rmlr.pdf}
    \vspace{-15pt}
   \caption{Huber regression generalises less well than MLR.}
    \label{fig:A}
\end{figure}
\end{minipage}
\end{frame}


\section{So it seems that in vulnerable places,
         a  majority of the variance can be predicted by $\tau_u$, and $\tau_v$.}
    \begin{frame}[plain]
        \vfill
      \centering
      \begin{beamercolorbox}[sep=8pt,center,shadow=true,rounded=true]{title}
        \usebeamerfont{title}\insertsectionhead\par%
        \color{oxfordblue}\noindent\rule{10cm}{1pt} \\
        \begin{itemize}
        \item Huber regression (supposedly more robust) does not significantly change the results.
        \item Approximately the same linear model is learnt, independent of year trained on.
        \item Miami, and the point near New Orleans stick out as places of poor fit.
        \end{itemize}
      \end{beamercolorbox}
      \vfill
  \end{frame}


  \begin{frame}{Orientation of the Regression Line}
\vspace{-30pt}
\hspace{-30pt}
 \begin{minipage}{1.1\textwidth}
 \begin{minipage}{0.7\textwidth}
\begin{figure}
   \hspace{-40pt} \includegraphics[width=1\linewidth]{../surge/plots/reg_angle.png}
    \vspace{-15pt}
   \caption{\texttt{np.arctan2(c0, c1)}}
    \label{fig:A}
\end{figure}
\end{minipage}
 \begin{minipage}{0.28\textwidth}
\begin{figure}[htb!]
        \vspace{-25pt}
   \hspace{-40pt} \includegraphics[width=0.95\linewidth]{../surge/plots/reg_angle_correlate.png}\\
    \hspace{-40pt} \includegraphics[width=0.95\linewidth]{../surge/plots/reg_angle_rsquare.png}
\end{figure}
\end{minipage}
\end{minipage}
\end{frame}

\section{There is a high correlation between the direction of the MLR and the
 coastline, but a specific $\sigma$ value is not picked out. }
    \begin{frame}[plain]
        \vfill
      \centering
      \begin{beamercolorbox}[sep=8pt,center,shadow=true,rounded=true]{title}
        \usebeamerfont{title}\insertsectionhead\par%
        \color{oxfordblue}\noindent\rule{10cm}{1pt} \\
        \begin{itemize}
        \item This is very gratifying, but leaves us with some problem.
        \item Increasing above $\sigma=5$ significantly increases correlation,
         suggesting that the particular orientation of the cell is not as important.
        \item The angle arrived at was also consistent between years.
        \end{itemize}
      \end{beamercolorbox}
      \vfill
  \end{frame}

\begin{frame}{The Magnitude of the  Regression Gradient.}
\vspace{-30pt}
    %\centering
\hspace{-30pt}
 \begin{minipage}{1.1\textwidth}
\begin{figure}[htb!]
    \centering
   \hspace{-40pt} \includegraphics[width=1.0\linewidth]{../surge/plots/reg_mag.pdf}
    \vspace{-15pt}
   \caption{\texttt{np.square(np.sqrt(c0) + np.sqrt(c1))}.
    This does not take account of the fact that the fits for some points are much better than others.}
    \label{fig:A}
\end{figure}
\end{minipage}
\end{frame}

\begin{frame}{Adjusted Regression Gradient.}
\vspace{-30pt}
    %\centering
\hspace{-30pt}
 \begin{minipage}{1.1\textwidth}
\begin{figure}[htb!]
    \centering
   \hspace{-40pt} \includegraphics[width=1.0\linewidth]{../surge/plots/adj_reg_mag.pdf}
    \vspace{-15pt}
   \caption{($\bar{r^2}$)\texttt{*np.square(np.sqrt(c0) + np.sqrt(c1))}. }
    \label{fig:A}
\end{figure}
\end{minipage}
\end{frame}

\begin{frame}{A reasonable $r^2$ can be produced on each.}
\vspace{-15pt}
    %\centering
\hspace{-30pt}
 \begin{minipage}{1.0\textwidth}
\begin{figure}[htb!]
    \centering
    \includegraphics[width=0.7\linewidth]{../surge/plots/ridge_lasso.pdf}
    \vspace{-15pt}
   \caption{Although not for lasso, and the regression is not consistent.
            We also need another coast to see if this generalises.}
    \label{fig:A}
\end{figure}
\end{minipage}
\end{frame}


\section{Linear Regression Algorithms (sort of) Work. }
\begin{frame}[plain]
        \vfill
      \centering
      \begin{beamercolorbox}[sep=8pt,center,shadow=true,rounded=true]{title}
        \usebeamerfont{title}\insertsectionhead\par%
        \color{oxfordblue}\noindent\rule{10cm}{1pt} \\
        \begin{itemize}
        \item Multicolinearity seems a problem in final plot.
        \item Other linear algorithms (e.g. RANSAC) could be tried
              to deal with non-normal regression.
        \item More isobath depths can be added, but this might make
              the algorithm just overfit more.
        \item Perhaps we now download the Chinese Coast from the same
              data set to see if it generalises.
        \end{itemize}
      \end{beamercolorbox}
      \vfill
\end{frame}


\section{Extreme Value Theory. }
\begin{frame}[plain]
        \vfill
      \centering
      \begin{beamercolorbox}[sep=8pt,center,shadow=true,rounded=true]{title}
        \usebeamerfont{title}\insertsectionhead\par%
        \color{oxfordblue}\noindent\rule{10cm}{1pt} \\
        \begin{itemize}
        \item Normal Choices are POT or Block Maxima, but these strategies require
              large amounts of data to converge~\cite{taleb2019much}.
        \item \texttt{skextreme}~\cite{skextremes} is an python package that implements these
              standard strategies.
        \item Taleb does present some solutions~\cite{taleb2019statistical}.
        \item As physical processes with understood causes, TCs
              and their storm surges are grey not black swans.
        \item We could find bias between hourly to daily output.
        \end{itemize}
      \end{beamercolorbox}
      \vfill
\end{frame}


\begin{frame}{Summary}
\begin{itemize}
\item The picture revealed by our current results broadly agrees well with that
      of~\cite{ZannaPreprint}, despite the current model being twenty four times
      the temporal resolution, and three times the lateral resolution.
%\item However each year of data does not fully sample the distribution
       (I mean that there can't be the expected fractional number of storms
        in a location in a year), but there may be ways around this.
\item As the coast is a self-similar fractal~\cite{mandelbrot1967long, richardson1961problem},
      it is unclear which length scale to average over to find summary statistics
      for a point on the coast. A reasonable solution would be to give the
      regression algorithm a number of these, and make it decide what is relevant.
\end{itemize}

\end{frame}

\begin{frame}{Current Priorities}
\begin{itemize}
\item Refine measure of the responsiveness  (storm efficiency in~\cite{ZannaPreprint})
      of a coastline unit to a wind stress event.
\item Try to untangle the contribution of
	    convexity, and bathymetry to the responsiveness.
\end{itemize}
\end{frame}

\begin{frame}{Thank you for listening!}
      \begin{minipage}{1.1\textwidth}
   \hspace{-20pt}\begin{minipage}{0.45\textwidth}
   \textbf{Resources Used:}
   \begin{itemize}
\item \texttt{matplotlib}~\cite{Hunter:2007} for original figures,
\item WebPlotDigitiser~\cite{WebPlotDigitiser} for data extraction,
\item Mathpix~\cite{mathpix} for maths extraction,
\item sci-kit-learn~\cite{scikit-learn}  for ML,
\item \texttt{cmocean}~\cite{thyng2016true}
for cmaps,
\item \texttt{numba.jit}~\cite{lam2015numba} for speed,
\item \texttt{uncertainties}~\cite{lebigot2010uncertainties} for error propogation,
\item \texttt{xarray}~\cite{hoyer2017xarray} for ND~data.
\end{itemize}
    \end{minipage}\hspace{10pt}
      \begin{minipage}{0.50\textwidth}
    \begin{figure}
            \includegraphics[width=1\linewidth]{images/example-images/Compared_O2.pdf}
            \normalsize{
    %$\tau=5.71\pm0.15\;\mathrm{yrs},\quad$\\
    $\implies \mathrm{t}_{\mathrm{double}}=4.0\pm0.1\;\mathrm{yrs}$}
                \caption{The number of papers ($\mathrm{N}$) with keyword ML
                 has increased exponentially over the last 25 years.\\ \textit{
                   Data Source: Web of Science.} %~\cite{WOS}
                   }
    \end{figure}
    \end{minipage}
\end{minipage}
\end{frame}


\begin{frame}[allowframebreaks]
\renewcommand*{\bibfont}{\footnotesize}
\footnotesize
        \frametitle{References}
        \printbibliography
\end{frame}

%%%%%%%%%%%%%%%%%%%%%%%%%%%%%%%%%%%%
\appendix
\normalsize
\section{Tauuo, $\tau_u$ }
    \begin{frame}[plain]
        \vfill
      \centering
      \begin{beamercolorbox}[sep=8pt,center,shadow=true,rounded=true]{title}
        \usebeamerfont{title}\insertsectionhead\par%
        \color{oxfordblue}\noindent\rule{10cm}{1pt} \\
                \includegraphics[width=0.93\linewidth]{images/tauuo.png}
      \end{beamercolorbox}
      \vfill
  \end{frame}

\begin{frame}{Yearly $\tau_u$ distributions are similar between 2004/5 }
\vspace{-20pt}
\begin{figure}[htb!]
    \centering
    \includegraphics[width=0.6\linewidth]{../surge/plots/tauuo/stats_points_plot.pdf}
     \hspace{0pt} \includegraphics[width=0.285\linewidth]{../surge/plots/tauuo/stats_points_plot_1.pdf}
    \vspace{-7pt}
    \caption{A comparison between the distributions of $\tau_u$ above geoid (tauuo)
     for the training set (2005) and the test set (2004).}
    \label{fig:}
\end{figure}
\end{frame}


\begin{frame}{No clear difference between 2004/5  }
\vspace{-20pt}
\begin{figure}[htb!]
    \centering
    \includegraphics[width=0.95\linewidth]{../surge/plots/tauuo/stats_points_plot_2.pdf}
    \vspace{-7pt}
    \caption{$\tau_u$.}
    \label{fig:}
\end{figure}
\end{frame}

\begin{frame}{$\tau_u$ }
\vspace{-20pt}
\begin{figure}[htb!]
    \centering
    \includegraphics[width=0.95\linewidth]{../surge/plots/tauuo/stats_points_plot_5.pdf}
    \vspace{-7pt}
    \caption{$\tau_u$ There does not seem to be anything wrong with the plots.}
    \label{fig:A}
\end{figure}
\end{frame}


\begin{frame}{$\tau_u$  Covariance and Correlation matrices are similar.  }
\vspace{-20pt}
\begin{figure}[htb!]
    \centering
    \hspace{-10pt}
    \includegraphics[width=0.68\linewidth]{../surge/plots/tauuo/corr_cov.pdf}
     \includegraphics[width=0.16\linewidth]{../surge/plots/tauuo/corr_cov_cbar.pdf}
    \vspace{-7pt}
    \caption{A/B: Covariance matrices for 2004/5.\\
    $\quad\quad\quad\;\;$C/D: Correlation matrices for 2004/5.}
    \label{fig:}
\end{figure}
\end{frame}

% tauvo
\section{Tauvo, $\tau_v$ }
    \begin{frame}[plain]
        \vfill
      \centering
      \begin{beamercolorbox}[sep=8pt,center,shadow=true,rounded=true]{title}
        \usebeamerfont{title}\insertsectionhead\par%
        \color{oxfordblue}\noindent\rule{10cm}{1pt} \\
                \includegraphics[width=0.93\linewidth]{images/example-images/tauvo.png}
      \end{beamercolorbox}
      \vfill
  \end{frame}

\begin{frame}{$\tau_v$ between 2004/5 }
\vspace{-20pt}
\begin{figure}[htb!]
    \centering
    \includegraphics[width=0.6\linewidth]{../surge/plots/tauvo/stats_points_plot.pdf}
     \hspace{0pt} \includegraphics[width=0.285\linewidth]{../surge/plots/tauvo/stats_points_plot_1.pdf}
    \vspace{-7pt}
    \caption{A comparison between the distributions for
             the training set (2005) and the test set (2004).}
    \label{fig:}
\end{figure}
\end{frame}


\begin{frame}{$\tau_v$  No clear discontinuity between 2004/5  }
\vspace{-20pt}
\begin{figure}[htb!]
    \centering
    \includegraphics[width=0.95\linewidth]{../surge/plots/tauvo/stats_points_plot_2.pdf}
    \vspace{-7pt}
    \caption{}
    \label{fig:}
\end{figure}
\end{frame}

\begin{frame}{$\tau_v$ Mean of Points. }
\vspace{-20pt}
\begin{figure}[htb!]
    \centering
    \includegraphics[width=0.95\linewidth]{../surge/plots/tauvo/stats_points_plot_5.pdf}
    \vspace{-7pt}
    \caption{}
    \label{fig:A}
\end{figure}
\end{frame}


\begin{frame}{$\tau_v$ Covariance and Correlation matrices are similar.  }
\vspace{-20pt}
\begin{figure}[htb!]
    \centering
    \hspace{-10pt}
    \includegraphics[width=0.68\linewidth]{../surge/plots/tauvo/corr_cov.pdf}
    \includegraphics[width=0.16\linewidth]{../surge/plots/tauvo/corr_cov_cbar.pdf}
    \vspace{-7pt}
    \caption{A/B: Covariance matrices for 2004/5.\\
            $\quad\quad\quad\;\;$C/D: Correlation matrices for 2004/5.}
    \label{fig:}
\end{figure}
\end{frame}

\begin{frame}{Running average period controls convexity measure }
\vspace{-20pt}
\begin{figure}[htb!]
    \centering
    \includegraphics[width=1.0\linewidth]{../surge/plots/angle_deriv.pdf}
    \caption{A 50 point running average may be a useful compromise. }
    % \label{fig:}
\end{figure}
\end{frame}


\begin{frame}{The change in angles along the coastline}
\vspace{-20pt}
\begin{figure}[htb!]
    \centering
    \begin{equation}
\bar{B_i}=\operatorname{arctan} 2\left(\frac{1}{n} \cdot
\sum_{j=i-\lambda}^{j=i+\lambda}  \sin B_{j},\; \frac{1}{n}
 \cdot \sum_{j=i-\lambda}^{j=i+\lambda} \cos B_{j}\right)
\end{equation}
    \includegraphics[width=1.0\linewidth]{../surge/plots/angles_plots.pdf}
    \caption{The angle along the coast can be calculated, and averaged in a couple
     of different ways. The average range is subjective.}
    % \label{fig:}
\end{figure}
\end{frame}

\begin{frame}{Let's see if we can remove sinusoids from the data. }
\begin{equation}
\Delta\eta_i = \eta_i - \bar{\eta_i} \sim a\cdot \sin{(\omega(t + \phi))}
\end{equation}
\vspace{-20pt}
\begin{figure}[htb!]
    \centering
    \hspace{-10pt}
    \includegraphics[width=1.1\linewidth]{../surge/plots/fits/coefficients_of_sine.pdf}
    \vspace{-7pt}
   \caption{Fitting sinusoids to each point along the coast for the years.}
    \label{fig:}
\end{figure}
\end{frame}



\begin{frame}{An Example around Miami}
\vspace{-30pt}
\begin{figure}[htb!]
    \centering
    \includegraphics[width=0.75\linewidth]{../surge/plots/miami_map.pdf}
\end{figure}
\end{frame}


\begin{frame}{An Example around Boston}
\vspace{-30pt}
\begin{figure}[htb!]
    \centering
    \includegraphics[width=0.75\linewidth]{../surge/plots/boston_map.pdf}
\end{figure}
\end{frame}


\section{Initial Responsiveness Regression Results\\
(Inspired by 10c \& 10d in Yin et al.~2020~\cite{ZannaPreprint})}
    \begin{frame}[plain]
        \vfill
      \centering
      \begin{beamercolorbox}[sep=8pt,center,shadow=true,rounded=true]{title}
        \usebeamerfont{title}\insertsectionhead\par%
        \color{oxfordblue}\noindent\rule{10cm}{1pt} \\
        \includegraphics[width=1\linewidth]{images/example-images/yin-responsiveness.png}
      \end{beamercolorbox}
      \vfill
  \end{frame}

\begin{frame}{Thresholding can radically change results  }
\vspace{-20pt}
\begin{figure}[htb!]
    \centering
    \hspace{-10pt}
    \includegraphics[width=0.6\linewidth]{../surge/plots/threshold_plot_new.pdf}
    \vspace{-7pt}
   \caption{Requiring  $\Delta \eta \ge 0$~m is a reasonable criterion,
    and almost maximises the correlation between the wind stress (no location information),
     and threshold at an arbitrary point.}
\end{figure}
\end{frame}

\begin{frame}{Linear Regression of $ \Delta \eta$ against $|U|^2$ for $ \Delta\eta>0$.  }
\vspace{-20pt}
\begin{figure}[htb!]
    \centering
    \hspace{-10pt}
    \includegraphics[width=0.9\linewidth]{../surge/plots/score-plot/score_plot.pdf}
    \vspace{-7pt}
   \caption{Only generalises for the most vulnerable points. Most variance not modelled.}
    \label{fig:}
\end{figure}
\end{frame}

\begin{frame}{Angle derivative $\sigma=15\pm2$ most correlated.   }
\vspace{-20pt}
\begin{figure}[htb!]
    \centering
    \hspace{-10pt}
    \includegraphics[width=1\linewidth]{../surge/plots/check_new.png}
    \vspace{-7pt}
    \caption{$r_p$ with angle derivative against A: training score,
    B: test score, C: linear reg coeff,
    D: $r_p$ of training data.}
    \label{fig:}
\end{figure}
\end{frame}

\begin{frame}{Extracting Bathymetry}
\vspace{-30pt}
\begin{figure}[htb!]
    \centering
    \includegraphics[width=\linewidth]{../surge/plots/simple_bath_extract.pdf}
\end{figure}
\end{frame}

\begin{frame}{Performing Simple Regression }
\vspace{-20pt}
\begin{figure}[htb!]
    \centering
    \hspace{-10pt}
    \includegraphics[width=1\linewidth]{../surge/plots/reg_ready.pdf}
    \vspace{-7pt}
    \label{fig:}
\end{figure}
\end{frame}

\begin{frame}{MLR }
\vspace{-20pt}
\begin{figure}[htb!]
    \centering
    \hspace{-10pt}
    \includegraphics[width=1\linewidth]{../surge/reg_look.pdf}
    \vspace{-7pt}
   \caption{$r_p =0.29$, $r_p =-0.48$, &  $r_p=0.06$. MLR r$^2$=0.308.   }
    \label{fig:}
\end{figure}
\end{frame}


\begin{frame}{$\sigma$ choice}
\vspace{-20pt}
\begin{itemize}
\item Great stuff. Is $\sigma = 10$ a good value?
% \item Chosen based on kurtosis plots (see later) by an NNN (me).
\end{itemize}
\vspace{-10pt}
% 'check_similarity'
\begin{figure}[htb!]
    \centering
    \includegraphics[width=1\linewidth]{../surge/plots/check_similarity.pdf}
   \caption{A: Mean, B: Std Dev, C: Skew, D: Kurtosis.}
    % \label{fig:}
\end{figure}
\vspace{-20pt}
\begin{itemize}
\item Maybe. $\sigma=15\pm2$ leads to the highest $r_p$ against
      the points standard deviation for the year.
\end{itemize}
\end{frame}


\end{document}

##########################------   END  -------#################################

\section{Code}
    \begin{frame}[plain]
        \vfill
      \centering
      \begin{beamercolorbox}[sep=8pt,center,shadow=true,rounded=true]{title}
        \usebeamerfont{title}\insertsectionhead\par%
        \color{oxfordblue}\noindent\rule{10cm}{1pt} \\
        \LARGE{\faFileCodeO}
      \end{beamercolorbox}
      \vfill
  \end{frame}

\subsection{Example}
\begin{frame}[fragile]{Example}
\begin{block}{Greatest Common Divisor}
\begin{lstlisting}[firstnumber=1, label=glabels, xleftmargin=10pt]
def greatest_c_remainder(a,b):
	'''Greatest common divisor of a and b'''
	r = a % b
	if r == 0:
		return b
	else:
		m = b
		n = r
	return greatest_c_remainder(m,n)

\end{lstlisting}
\end{block}
\end{frame}

\begin{frame}{Sobel Edge Detection %~\cite{hjelmervik2019detection}
}
\only<1>{

\vspace{-20pt}\begin{equation}
    \underline{\underline{{G}}}_{{x d}}=\left[\begin{array}{ccc}{1} & {0} & {-1} \\
                                                        					  {2} & {0} & {-2} \\
                                                       					   {1} & {0} & {-1}
                                      				  \end{array}\right]
                             						* \underline{\underline{X}}_{d}
\end{equation}

\begin{equation}
    \underline{\underline{{G}}}_{{y} d }=\left[\begin{array}{ccc}{1} & {2} & {1} \\
                                                     						{0} & {0} & {0} \\
                                                     						{-1} & {-2} & {-1}
                                            					 \end{array}\right]
                                        				* \underline{\underline{X}}_{d}
\end{equation}

Can combine these together to give a single number for the strength of the edge in
the principal component at that point:

\begin{equation}
    {G}^n_{{d}}=
    \sqrt{{{G}^n_{{x}{ d}}}^{2}
        + {{G}^n_{{y  d}}}^{2}}.
\end{equation}

To scale this the value of \(\underline{\underline{{G}}}_{{d}}\) can be expressed as some number
    of \(\sigma\) above the mean, although \({G}^n_{{d}}>0\) and not normal.}
\only<2>{
Deleting un-needed formula.
}
\end{frame}


\section{Equations}
    \begin{frame}[plain]
        \vfill
      \centering
      \begin{beamercolorbox}[sep=8pt,center,shadow=true,rounded=true]{title}
        \usebeamerfont{title}\insertsectionhead\par%
        \color{oxfordblue}\noindent\rule{10cm}{1pt} \\
        \LARGE{\faFileTextO}
      \end{beamercolorbox}
      \vfill
  \end{frame}


  { % all template changes are local to this group.
    \setbeamertemplate{navigation symbols}{}
    \begin{frame}<article:0>[plain]

        \begin{tikzpicture}[remember picture,overlay]
            \node[at=(current page.center)] {
                \includegraphics[keepaspectratio,
                                 width=\paperwidth,
                                 % height=\paperheight
                                 ]{example-images/cambridge-surge.png}
            };
        \end{tikzpicture}

          %\url{https://coastal.climatecentral.org/map/6/}
     \end{frame}
}

{ % all template changes are local to this group.
    \setbeamertemplate{navigation symbols}{}
    \begin{frame}<article:0>[plain]
        \begin{tikzpicture}[remember picture,overlay]
            \node[at=(current page.center)] {
                \includegraphics[keepaspectratio,
                                             width=\paperwidth,
                                            % height=\paperheight
                                            ]{example-images/new-orleans-surge.png}

            };
        \end{tikzpicture}
                  %\url{https://coastal.climatecentral.org/map/6/}
     \end{frame}
}


%\section*{Outline}\begin{frame}{Outline}\tableofcontents\end{frame}

% \section{Text}
%    \begin{frame}[plain]
%        \vfill
%      \centering
%      \begin{beamercolorbox}[sep=8pt,center,shadow=true,rounded=true]{title}
%        \usebeamerfont{title}\insertsectionhead\par%
%        \color{oxfordblue}\noindent\rule{10cm}{1pt} \\
%        \LARGE{\faFileTextO}
%      \end{beamercolorbox}
%      \vfill
%  \end{frame}
