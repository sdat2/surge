\section{Repositories}
To reproduce the work in this report, it should be possible to just run these
commands in terminal:

\begin{verbatim}
> git clone https://gitlab.com/sdat2/surge.git

> git clone https://github.com/sdat2/scikit-extremes.git

\end{verbatim}

and then follow the instructions in the \texttt{surge} README file.

\subsection{\texttt{surge} repository}

My primary code git repository is called `\texttt{surge}'~\cite{gitlab}.
It is quite large (see below), and I do not think that it would be helpful to reproduce any of the code here
as the logical structure of any interesting code is represented by the
algorithms included in the report. There are so many lines of code within the file
repository as shown below, and I think the easiest way to understand it would be to open
and run the repository in an appropriate development environment rather than a pdf file.
The repository comes with the slimmed down point data files for the american and
viet-chinese coast, and all of the results in this report could be reproduced from them.
This repository was developed from scratch by myself during this academic year 2019-20.
There are a couple of files that I borrowed from others who were
experienced with using the HPC (e.g. bash batch job file).


\verbatiminput{../surge/environment/cloc.txt}

Below is the README of the \texttt{surge} repository which is designed to
introduce the project and give instructions to replicate the results:

\verbatiminput{../surge/README.md}

\subsection{\texttt{scikit-extremes} repository}

To implement models from univariate extreme value theory, I built
on an unfinished repository that I locally installed from github,
called \texttt{scikit-extremes}~\cite{skextremes}.

\verbatiminput{../scikit-extremes/cloc.txt}

\subsection{\texttt{report} repository}

I wrote the report and presentations in \LaTeX.

\verbatiminput{appendices/cloc.txt}
