\twocolumn[
  \begin{@twocolumnfalse}
  \maketitle

  \vspace{-15pt}
  \setcounter{page}{1}
    \begin{abstract}
    Storm surges are hazardous for life and property.
     I propose a new set of methods whereby they can be studied
     within an arbitrary climate model without inundation;
     I introduce simple algorithms for selecting coastlines,
      with two new metrics for the bathymetry and convexity of the coast.
       I create two new metrics for the responsiveness of a coastline point to the sea stress state.
        The bathymetry and convexity fail to be robustly predictive of responsiveness coastlines
        in an ORCA12 NEMO hindcast of 2004-5, but I hope that these can be developed in future.
        Finally I use the potential intensity theory of hurricanes, combined with extreme value theory,
        for a 101-year experiment (\texttt{control-1950}, ORCA12, HadGEM3)
        to show that the there is a reasonable potential method for combining physical
        and statistical knowledge to better estimate hazards,
        although this requires further testing. This should provide
        a strong basis for future work applying these
        methods to the ensemble of global climate models,
        and incorporating observational data.

\vspace{10pt}
    \footnotesize{$^{*}$\thanksmessage}

    \vspace{-10pt}
    \normalsize


    % \paragraph{Keywords:} \textit{Tropical Cyclones (TCs),
    %                              Physically Limited Extreme Value Theory (EVT),
    %                              Warped Gaussian Processes (GPs),
    %                              Machine Learning (ML).
    %                              }
    \end{abstract}
    \vspace{20pt}
  \end{@twocolumnfalse}
]
