
``Cyclogenesis is one of the great
 mysteries of the tropical atmosphere''~\cite{emanuel2018progress};
although there is thermodynamic disequilibrium between the tropical sea
surface and the atmosphere TCs cannot spontaneously emerge. As first shown by
Gray 1975~\cite{gray1975tropical} there are
 diverse phenomena that triggered the growth of TC:

 In the North Atlantic, the primary triggers are African easterly waves (AEWs),
 that sometimes deepen into tropical depressions and then hurricanes,
 although AEWs can be poorly resolved in models~\cite{tomassini2017interaction}.
 Cyclogenesis can also be caused by a cold front that penetrates the tropics.


Bister~\&~Emanuel~2002~\cite{bister2002low} provides an algorithm whereby given the rest of the climate
there is some maximum intensity that a hurricane could reach~\cite{bister2002low}
as a product of a development of Emanuel's work~\cite{bister1996development,
bister1998dissipative, bister2002low}


\subsection{Contrast to extratropical cyclones}
In contrast to TCs,  ECs do not need a trigger, forming
spontaneously through the baroclinic instability
in the extratropical atmosphere~\cite{lorenz1960energy}.
ECs seem to come in a continuous spectrum.
Given that they cause
significantly less damage, and no theory has been
 set out for calculating their maximum size,
they will be ignored in the rest of this thesis.
