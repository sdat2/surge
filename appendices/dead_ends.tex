
\subsection{Dead Ends}

\begin{quote}
``I have not failed. I've just found 10,000 ways that won't work.''
Thomas Edison
\end{quote}

Initially I was quite convinced that it would be relatively
simple to apply the algorithms that my peers where applying
(Gaussian Processes) to this new problem~\cite{LeMaitre2019gaussian}, but I did not
forsee all of the technical problems that would be in the way of that.
Although it would have been possible to regress $\tau_u$, $\tau_v$ against
$\eta$ using Gaussian processes, and this would have been bound to produce a
higher $r^2$ value than linear regression because of the increased number of
degrees of freedom, it would have been impossible to give all of the points,
there is no simple way of summarising what the kernel contains,
and it would have taken many hours on the HPC to process the coastline.

Gaussian processes do seem a particularly promising route to go down
with regards to hybrid ML/physical modelling, as the some degree of knowledge
of the physical system can and should be encoded in the kernel~\cite{duvenaud2014automatic}.
However, in the context of extreme events, it would be wise to think more carefully
before applying this technique.

Another influence to  my de-prioritisation of this
avenue was reading Taleb's \textit{The Black Swan}~\cite{taleb2007black} during Lent
term, which included extensive criticism of using Gaussian based methods in the
real world, particularly for the estimation of risk.
I also attended a Gaussian Processes Cambridge meetup where a practitioner noted that
Gaussian processes do not naturally cope well with extreme
events. This lead me to T19~\cite{taleb2019statistical} and using extreme value
theory with a greater amount of data using \texttt{control-1950}.
It does not rule out that using warped Gaussian processes, approximated as
warped SDEs, will not patch over these deficiencies in the future.

When I initially began to look at the ORCA12 output,
I thought that it would be easier to go from that 2D
curvilinear grid, to a rectilinear lat-lon grid before
I began to interpret the data. It was not obvious how
this was to be done, and so I asked Dr.~Mathiot, who recommended
I use SOSIE \texttt{fortran} package for interpolation.\footnote{\url{https://github.com/sdat2/sosie}}
The package was eventually coerced into working on my laptop and the JASMIN
HPC after I had run some example \texttt{fortran}
code\footnote{\url{https://gitlab.com/sdat2/fortran-examples}} on my laptop
to get my head round the language. It was only after this introduction to
the new language that I realised that interpolating the results was
unnecessary, and that it would probably lead to artefacts in the product as coastal
locations inherit the mask of the adjacent land.

I used \texttt{iris} to verify that the deformation of the ORCA12 grid was quite small
for all of the points that I initially chose (from the East~US coastline).

\paragraph{The MOOSE problems}

The \texttt{two-year} run was stored on the
Met Office's tape's.
To access the \texttt{two-year} run I used the
MASS transfer system on the JASMIN HPC.
During the first week and a half this system broke
with no restart after about 4GB of file transfer,
which meant that only the smallest files could be
moved to the group workspace. This seemed to magically resolve
itself the night before my second project meeting.
Originally, I was also not using \texttt{screen}
and therefore there was the additional failure-mode
that the wifi access from BAS to JASMIN would break,
thus halting the transfers to the group work space,
with no way to restart, except from the beginning.
There was also no way to take only a subset of the file.

\paragraph{Interpretation of results}
As shown in our first order thermodynamic understanding, it seems likely that
global warming will lead to an increase in hurricane intensity, independent
of the imperfections of the current round of global climate models,
but instead rests on simple physical argument (see §~\ref{sec:1d-hurr}).
Even if the most severe outcomes may not occur,
it would be as imprudent to not to take mitigating action as it would be
to not buy fire insurance for your house~\cite{emanuel2005divine};
public policy is based on the worst case event rather than the average~\cite{taleb2020statistical}.
