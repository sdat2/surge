\section{Introduction}
\label{sec:1_Introduction}


Storm surges are coastal sea levels which far exceed those
expected given the tide.
They are caused by tropical cyclones
(TCs) and extratropical cyclones (ECs).
TCs draw their kinetic energy from the thermodynamic disequilibrium between
the tropical sea surface and the troposphere~\cite{emanuel1986air, emanuel1987dependence},
whereas ECs draw theirs from the
polewards temperature gradient~\cite{lorenz1960energy, holton2004introduction}\footnote{
This is technically called baroclinic instability.} at higher latitudes.\footnote{
Practically they are differentiated using algorithms like \texttt{TSTORMS}
\url{https://www.gfdl.noaa.gov/tstorms/} as in Y20~\cite{ZannaPreprint}.}
Because they have lower central pressures and sustain greater winds,
TCs create more dangerous storm surges~\cite{emanuel2005divine},
and so ECs are largely ignored in this thesis.

TCs have historically been the
world's largest physical hazard~\cite{shultz2005epidemiology},
for economic damage,
and specifically TC storm surges for lives lost~\cite{shultz2005epidemiology,
zhang2009tropical, emanuel2005divine}.
This hazard is increasing as more people live on vulnerable coastlines
in substandard buildings~\cite{emanuel2005divine}.
Anthropogenic climate change is expected to exacerbate this,
as it begins to increase the maximum TC intensity\footnote{
However the uptick in TCs is not currently above
 natural variability~\cite{mendelsohn2012impact}.}
(see~§~\ref{sec:hurr-theory})~\cite{emanuel2008hurricanes,emanuel2017will, nordhaus2010},
and to sea level rise placing more areas at risk~\cite{SROCC}.
The range of TCs may also increase~\cite{emanuel2008hurricanes,
emanuel2017will, fedorov2010tropical, }.

Quantifying how this hazard will change is necessary
so that the public can act with foresight.
Others have already begun to apply machine learning (ML) techniques to this
end~\cite{kulp2019new, kulp2018coastaldem, tadesse2020data}.
The motivation for hybrid physical/ML modelling is simple:
an algorithm that does not respect physical laws
may fit the data well, \textit{but it is incorrect};
it is more likely to generalise poorly to unseen data,
and \textit{fail dangerously}~\cite{beucler2019achieving}.


\begin{table}[ht]
\resizebox{\columnwidth}{!}{%
\begin{tabular}{l L{6.5cm}}

\hline \hline
\textbf{Ab} & \textbf{Expansion}\\
\hline
EVT & Extreme value theory \\
GEV & Generalised extreme value (distribution) \\
BM & Block maxima \\
ML & Machine learning \\
GP & Gaussian process \\
TC & Tropical cyclone \\
EC & Extratropical cyclone \\
CMIP & Coupled model intercomparison project \\
Y20 & Yin et al.~2020~\cite{ZannaPreprint} \\
T19 & Taleb 2019, \textit{The Statistical Consequences of Fat Tails}~\cite{taleb2019statistical}\\
SSH & Sea surface height above geoid (m) \\
SST & Sea surface temperature ($^{\circ}$C) \\
\texttt{tyr} & \texttt{two-year} 2004-5 hindcast (§~\ref{sec:rean-prod}) \\
\texttt{c50} & \texttt{control-1950} experiment (§~\ref{sec:control-1950-intro}) \\
\texttt{vc} & N-Vietnam \& China coastline (§~\ref{sec:coast-sample})\\
\texttt{eUS} & East US coastline (§~\ref{sec:coast-sample})\\
\hline \hline
\end{tabular}
}
\caption{Abbreviations (Ab) used in this report.}
\label{tab:abb}
\end{table}


This thesis begins to develop new methods of achieving this goal
as inspired by Yin~et~al.~2020~\cite{ZannaPreprint} (henceforth Y20).
Initially, by developing algorithms for extracting the coastlines from climate models
and metrics to characterise that coastline in §~\ref{sec:coast}.
Developing from Y20~\cite{ZannaPreprint}, I compare measures for the
responsiveness of a stretch of the coastline to wind stress.
I show using extreme value theory (EVT), that the
distributions of extreme events are controlled by the same factors as
the responsiveness under normal circumstances.
Finally, I suggest a
method of improving a naive EVT fit, using our physical knowledge
of the thermodynamic limit
to the maximum intensity of a hurricane~\cite{emanuel1999thermodynamic}
 given a climate (see~§~\ref{sec:hurr-theory})~\cite{emanuel1987dependence,
 emanuel2016predictability}.
I discuss the merits and deficiencies of these,
and suggest how they could be built upon in future work (§~\ref{sec:future}).
