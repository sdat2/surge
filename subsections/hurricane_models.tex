\subsection{Simple Tropical Cyclone Models}

\subsubsection{Introduction to azimuthally symmetric models}
One of the dominant controls on storm surge events on the Eastern Seaboard
 of the US are the large tropical cyclones (Hurricanes).
  These large vortices can be surprisingly well modelled by azimuthally
   symmetric analytic pressure distributions, and the velocity distribution
    that would be expected to result from them.

\subsubsection{Holland Hurricane Model}
The most prominent is the  Holland model~\cite{holland1980analytic,holland2010revised}
 which elaborated on a class of functions initially proposed
 by Schloemer 1954~\cite{schloemer1954analysis}.
  This model uses the dimensionless pressure \( P^{\prime}\) given by

\begin{equation}
    P^{\prime}=\dfrac{P-P_{c}}{P_{n}-P_{c}},
\end{equation}
where $P_c$ is the pressure at the centre of the storm, and $P_n$ is
the ambient pressure, ideally at at an infinite distance from the storm.
They then define the pressure relation
\begin{equation}
    r^{B}\ln(\frac{1}{P^{\prime}})=A,
\end{equation}or in a more elegant form
\begin{equation}
P^{\prime}=\exp{(-\frac{A}{r^B})},
\end{equation}
where A and B are model parameters which must be found empirically.

In steady state around this  pressure distribution generally yields a
 velocity~\cite{roisin2010GFD} of:
\begin{equation}
    U_g = \sqrt{AB(P_{n} - P_{c})\frac{\exp{(-\frac{A}{r^{B}})}}{\rho r^{B}}
     +\frac{r^2f^2}{4}}-\frac{rf}{2}.
\end{equation}
If we assume that the eye of the TC itself is small s.t. the Coriolis force
 is much smaller than the centrifugal then we can approximate this as

\begin{equation}
U_c = \sqrt{AB(P_{n}-P_{c})\frac{\exp{(-A/r^{B})}}{\rho r^{B}}}.
\end{equation}
The maximum velocity
\begin{equation}
U_m = (\frac{B}{\rho e})^{0.5}(P_n-P_c)^{0.5},
\end{equation}
which occurs at
\begin{equation}
R_m = A^{1/B}.
\end{equation}
There are at least two degrees of freedom:
B defines the height of the peak and A determines its relative position.
 The difference between the ambient and central pressure of the vortex still
 enter into the velocity expression, so there are four degrees of freedom when
  those are included.


\subsubsection{The Rankine Vortex}

The Rankine vortex is
\begin{equation}
C=\left\{\begin{array}{ll}U \frac{R_m}{ r}& \text{ if } r < R_m,\\
U\frac{r}{ R_m} & {\text { if } r>R_m},
\end{array}\right.
\end{equation}
where C is a constant, and the two sections are joined together
 so that the velocity is continuous at the boundary.
   The two degrees of freedom are $R_m$ and C.
This can be modified to a more general relation where $Ur^{x}=\text{constant}$
where X is a new parameter to fit,
which lies between 0.4 and 0.6~\cite{holland1980analytic}.
