\begin{figure}[htb!]
\centering
\includegraphics[width=\linewidth]{../surge/plots/point_choosing/standard_loc_test.pdf}

\vspace{-5pt}

\caption{The east coast of the United states, points chosen to be similar to
Y20~\cite{ZannaPreprint}.}
\label{fig:america}


\includegraphics[width=\linewidth]{../surge/plots/vin-china-choice.pdf}

\vspace{-9pt}

\caption{The northern coast of Vietnam (the Gulf of Tonkin)
and the mainland coastline of China, with the coastal cells chosen shown in green,
and the depths from ETOPO2v2.}
\label{fig:vin-china}
\end{figure}


\section{Coastlines within a climate model}
\label{sec:coast}
\subsection{Selection}
 \subsubsection{Fractal coastline}
 \label{sec:fractal}
 The coastline appears to be a self-similar fractal~\cite{mandelbrot1967long,
 richardson1961problem}. This means that it has the same features at all
 scales, so that the length of the coast increases without limit as resolution
 is increased, following the formula:
 \begin{equation}
 L(G)=M G^{1-D},
 \end{equation}
 where $L$ is the total length, $G$ is the length of a side,
 and $D>1$ is the fractal dimension~\cite{mandelbrot1967long}.
 A consequence of this is that a coastline has no defined length without
 reference to the resolution at which it is measured.
 This presents some problem when comparing different climate models with different resolutions.
 It has been observed that the fractal dimension changes
  along the East US coast ($1.11<D<1.20$ in one degree latitude bins)~\cite{jiang1998fractal}
 and this would mean that the relative number of coastal cells between different
 points will also change as the resolution is increased.
 This implies that the only practical way of labeling the points is by
 reference to certain reference locations as listed in
 Figures~\ref{fig:america}~\&~\ref{fig:vin-china}.

\subsubsection{Algorithms}
\label{sec:coast-algorithms}
Algorithms~\ref{algo:Selection}~\&~\ref{algo:Sort} first extract the coastline
from the surface slice of the ocean field and then pick out a single chain of
coastal points between two chosen ends. The list of points can be further
manually sorted to remove unwanted regions such as large bays
(e.g.~Chesapeake~bay)
and the occasional river island (e.g.~Shanghai bay's island). Once this is
done \texttt{xarray} can be used to quickly extract these points
from the climate model to make a new NetCDF file that contains all of the
required fields over a large number of timesteps for those points.



\begin{algorithm}[H]
\dontprintsemicolon
% \DontPrintSemicolon % Some LaTeX compilers require you to use \dontprintsemicolon instead
\caption{Coastal Selection}
\label{algo:Selection}
\KwIn{Grid: boolean grid (True for NaN values (land))}
\For{indices in Grid}{
  \If {Grid[indices] == False}{
     \If{ any of Grid[index\_x $\pm$ 1, index\_y $\pm$ 1] are True}
        {\textbf{append} indices to List\;}
     }
}
\KwOut{List: list of indices}
\end{algorithm}

\begin{algorithm}[H]
\dontprintsemicolon
% \DontPrintSemicolon % Some LaTeX compilers require you to use \dontprintsemicolon instead
\caption{Coastal Sort}
\label{algo:Sort}
\KwIn{Input\_List: Slimmed output of Algorithm~\ref{algo:Selection}}
\KwIn{First\_Index\_Pair: Point at one end of the coast }
\KwIn{Final\_Index\_Pair: Point at other end of the coast }

\texttt{tmp} $\gets$ First\_Index\_Pair\;

\Repeat{\texttt{tmp} = Final\_Index\_Pair}{
 \textbf{append} \texttt{tmp} to Output\_List\;
 \textbf{remove} \texttt{tmp} from Input\_List\;
 \texttt{tmp} $\gets$ Input\_List[\textbf{min}$_{k}$(\textbf{distance}(\texttt{tmp}, Input\_List[k]))]\;
}
\textbf{append} Final\_Index\_Pair to Output\_List\;

\KwOut{Output\_List: sorted list of indices}
\end{algorithm}


\subsubsection{Sample coastlines}
\label{sec:coast-sample}
The first coastline chosen was the eastern coastline of the mainland
United States, so that it could be compared with~\cite{ZannaPreprint},
and because the topography of the NEMO models is most accurate along the
US coastline (see Figure~\ref{fig:america}).
Secondly I extracted the coastline from Hue, Northern Vietnam, to China's
border with North Korea, removing Shanghai bay island (see Figure~\ref{fig:vin-china}).
Finally we extracted the eastern coastline of Honshu island, Japan.



\subsection{Metrics}
\subsubsection{Motivation}
A quick look at the coast revealed that certain features stuck out:
high kurtosis at New Orleans (NO), at the end of a shallowly sloping bay;
low kurtosis at Miami (MM), at the end of the headland of Florida.
But it would be interesting to find out what is really the controlling
factor, or whether either of these are truly predictive of storm surge risk.

To assess this question, it is necessary to quantify these two coastal attributes
and given that the coast is fractal (§~\ref{sec:fractal}), these metrics must also be defined at
all scales.

\subsubsection{Convexity}
\label{sec:convexity}
Given a selected coastline, we have a list of points of latitude and longitude.
We can define the normal bearing to the coast as in Equation~\ref{eq:bearing}.


%\begin{figure*}[htb!]
\begin{equation}
B_i^{\prime}=\operatorname{arctan} 2
\left(S_i,\; C_i\right)
\label{eq:bearing}
\end{equation}

with
\begin{equation}
S_i = \sum_{j=i-4\sigma}^{j=i+4\sigma} \mathcal{N}(i, \sigma^{2})\cdot \sin{( B_{j})},
\end{equation}
\begin{equation}
C_i  =  \sum_{j=i-4\sigma}^{j=i+4\sigma} \mathcal{N}(i, \sigma^{2})\cdot  \cos{( B_{j})},
\end{equation}

with reflective boundary conditions.

An example of this is shown in Figure~\ref{fig:angle_heatmap}} for the
East~US coastline. We can then define the symmetric derivative of this
smoothed angle,

\begin{equation}
\frac{d B_i^{\prime}}{d \mathrm{pt}}
\equiv \frac{B_{i+1}^{\prime}-B_{i-1}^{\prime}}{2},
\label{eq:bearing_derivative}
\end{equation}

where I apologise for a slight abuse of notation. This provides a convexity value for every smoothing length $\sigma$, as
shown in Figure~\ref{fig:derivative}A-C. The reflective boundary conditions
ensure that the bearing derivative is 0 at either end of the coastline.





\begin{figure}[htb!]
\centering
\includegraphics[width=\linewidth]{../surge/plots/angle_heatmap.pdf}
\caption{Normal bearing, $B$, along the east U.S.~coast.
         The Gulf of Mexico is visible as the curve between Rockport (RP)
         and Miami~(MM).
         }
\label{fig:angle_heatmap}

\includegraphics[width=\linewidth]{../surge/plots/derivative_heatmap.pdf}
\caption{Convexity metric along the east U.S.~coast.
         The three panels above show bay like concavity in blue, and convexity
         in red. At the largest $\sigma$ only the largest headlands such as
         Florida are visible (C), whereas the smallest bays are visible at a lower
        $\sigma$ (A).
}
\label{fig:derivative}
\end{figure}


\begin{figure}[htb!]
\includegraphics[width=\linewidth]{../surge/plots/bath_list.pdf}
\vspace{-25pt}

\caption{Isobaths plotted for the East US~coast.}
\label{fig:bath}
\includegraphics[width=\linewidth]{../surge/plots/distance_isobath.pdf}
\vspace{-25pt}

\caption{Distance to isobaths from points on East US~coast. MM is close to 500m
contour, as the bathymetry rapidly drops off, whereas at NO the drop off is more
gradual.}
\label{fig:isobath}
\includegraphics[width=\linewidth]{../surge/plots/isobath_correlate.pdf}
\vspace{-25pt}

\caption{The correlation matrix between the different isobath's distance's to
the East~US coast. High correlation between adjacent isobaths
suggests that multicolinearity will a problem for this metric~\cite{faul2019concise}.}
\label{fig:isobath}
\end{figure}


\subsubsection{Bathymetric gradient}
\label{sec:bath-grad}
The simplest way of working this out is to look at the
bathymetry netCDF file that the model was run with (ETOPO2v2-derived in this
 case (§~\ref{sec:nemo})), and work out where some set of chosen isobaths are.
Once these isobaths have been selected, the cartesian distance
within the model grid from every point on
the coast to that isobath are calculable.
This provides a distance-to-isobath value for each of the isobaths queried.

\FloatBarrier
