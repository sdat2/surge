\documentclass[usenames, dvipsnames]{article}      % article class
\usepackage[utf8]{inputenc}
\newcommand{\mytitle}{Hybrid tropical cyclone hazard modelling}
\newcommand{\penname}{Simon~Thomas}
\usepackage{moreverb}
\RequirePackage{fancyhdr}    % use fancyhdr to customize hdrs & ftrs
\RequirePackage[margin=2.5cm, headheight=15pt, includeheadfoot]{geometry}
\renewcommand{\headrulewidth}{0pt}   % suppress rule across top(fancyhdr default)
\setlength\headheight{26pt} %% just to make a warning go away.
\rhead{\thepage}  % right head page number.
% page number on right head
\lfoot{\penname}  % left foot author.
\cfoot{}
\rfoot{}
\pagestyle{fancy}  % use fancyhdr style

\usepackage{Theme/mystyle}
\usepackage{Theme/linkcolors}
\usepackage{Theme/referencing}
\usepackage{Theme/globals}
\graphicspath{{images/}{../images/}}
\addbibresource{references/generic_references.bib}
\addbibresource{references/references.bib}
\addbibresource{references/fluid_dynamics.bib}
\addbibresource{references/machine_learning.bib}
\addbibresource{references/global_warming.bib}
\addbibresource{references/programming.bib}
\addbibresource{references/surge.bib}
\addbibresource{references/tebbutt.bib}
\addbibresource{references/taleb.bib}

\addbibresource{phd.bib}
\renewcommand{\familydefault}{\sfdefault}

% \RequirePackage{fancyhdr}    % use fancyhdr to customize hdrs & ftrs
\RequirePackage[margin=2.5cm, headheight=15pt, includeheadfoot]{geometry}
\renewcommand{\headrulewidth}{0pt}   % suppress rule across top(fancyhdr default)
\setlength\headheight{26pt} %% just to make a warning go away.
\rhead{\thepage}  % right head page number.
% page number on right head
\lfoot{\penname}  % left foot author.
\cfoot{}
\rfoot{}
\pagestyle{fancy}  % use fancyhdr style

\RequirePackage[linesnumbered,ruled,vlined]{algorithm2e}

\usepackage{array}
\newcolumntype{L}[1]{>{\raggedright\let\newline\\\arraybackslash\hspace{0pt}}m{#1}}
\def\labelitemi{--}

\title{\vspace*{-100pt}\textbf{\mytitle}\vspace{-15pt}}
\date{\vspace{-10pt}\today \vspace{-15pt}}
\author{A Proposal for a PhD project 
at the AI4ER CDT program\\
Simon Thomas
}

\usepackage[fontsize=12pt]{fontsize}
\begin{document}
\maketitle

\vspace{-10pt}% \vspace{-5pt}

\footnotesize{
This proposal is the result of my own work and includes nothing which is the outcome of work done in collaboration, except where specifically indicated in the text and/or bibliography.}

\vspace{-10pt}

\subsection*{ Motivation}

Tropical cylones are heat engines 
running between the sea surface and the tropopause
(\cite{emanuel1986air, emanuel1987dependence}). 
The work that they produce create waves, 
and piles up the water into a mound at the centre of the storm. 
At landfall, this mound of water topped with waves creates
a storm surge, defined as a water level that 
far exceeds that expected given the tide.


Tropical cyclones have been the
largest physical hazard for economic 
damage~(\cite{Chavas2013U.S.Perspective}),
and their storm surges kill the most
people~(\cite{shultz2005epidemiology,Mayo2019PredictingResilience}).

A problem in this field is that the
complicated physical storm surge models that
are used are too expensive 
to produce a large number of extreme event outputs.
However, to model extreme event distributions rigorously
you need a large number of extreme
events~(\cite{taleb2019statistical}).
It would seem to be a better trade off to have larger 
amounts of less precise data so that you could reliably
fit the extreme value distribution.
To produce this data, we could use an ML model
which could reduce the reduce the 
computational complexity by 2 orders of magnitude
(e.g.~\cite{Guo2021Data-drivenNetworks}).
However, adding physical knowledge normally improves generalisability
(e.g.~\cite{beucler2019achieving}) and this is sometimes
known as \textbf{hybrid} physics/ML.
Implementing this is discussed below.

\vspace{-10pt}

\subsection*{Extreme value theory (3~months)}

Can we improve an extreme value fit by 
using more informative physical covariates?

\begin{itemize}
    \item Dataset: US East coast destruction as a result of historical
    Tropical Cyclones (normalised as \% of local property)
    \item Model: Generalised Pareto distribution for the events
    above a given threshold as in \cite{Chavas2013U.S.Perspective}.
    \item Measure of success: reduction of the size of the
    residuals between the empirical distribution and the model
    output distribution.
\end{itemize}

\vspace{-10pt}

\subsection*{Emulation of the ADCIRC/SWAN 
            storm surge model (8~months)}
Can we emulate an expensive storm surge model? How? Does a hybrid
physics/ML method work best?~\cite{Kashinath2021Physics-informedModelling, Guo2021Data-drivenNetworks}

\begin{itemize}
    \item Dataset: ADCIRC/SWAN runs of
    storm surges from Talea Mayo,
    and any additional runs of the model if this is insufficient
    to train the ML method.
    \begin{itemize}
        \item Split into training/test data.
    \end{itemize}
    \item Input: Sea surface wind vector, 
    bathymetry (depths at each point).
    \item Target: Sea surface height at each point in the domain.
    \item Model:
    \begin{itemize}
        \item A: Convolutional neural network (CNN) as in \cite{Guo2021Data-drivenNetworks}.
        %\begin{itemize}
        %    \item ReLu activation function as %in~\cite{beucler2019achieving}
        %\end{itemize}
        \item B: Hybrid model CNN (after A) with:
        \begin{itemize}
            \item EITHER: Penalty function for violation 
                of conservation of mass (e.g.~\cite{beucler2019achieving}).
            \item OR: Constraints added to the architecture
            (e.g.~\cite{beucler2019achieving}).
        \end{itemize}
        \item C: SLOSH model (a simpler physical baseline).
    \end{itemize}

    \item Measures of success: 
    \begin{itemize}
        \item RMSE between the prediction and the target for the
        test data.
        \item Computational cost in terms of seconds to produce prediction using test data.
    \end{itemize}
\end{itemize}

The setup in A~\&~B will be quite similar to
\cite{beucler2019achieving}, where we can see
whether the inclusion of physical constraints or penalties
improve the ability of the system to generalise.
Model C is there to provide another fair baseline.


\subsection*{Writing up first year report (1~month)}

Summarise the results of the first two stages.

\subsection*{Historical downscaling (9~months)}

Can whichever emulation method(s) has worked the best 
in the first year be used to downscale from 
model scale (a few km) to real surge size
at the coast?


\begin{itemize}
    \item Dataset: ECMWF ERA5 reanalysis
        (\cite{Zuo2019TheAssessment}) product 
        \& East US tidal gauges~(\cite{CO-OPSCurrents}), 
        selecting the more extreme events above 
        a threshold of windspeed.
    \item Input: Sea surface wind vector from ECMWF ERA5 reanalysis
    (\cite{Zuo2019TheAssessment}).
    \item Output: Sea surface 
        height at each sea surface point
        in the domain.
    \item Target: Sea surface height at the US tidal
        gauges~(\cite{CO-OPSCurrents}).
    \item Measure of success: RMSE between model prediction
        and the tidal gauge readings during events
        above the chosen threshold. 
\end{itemize}

\subsection*{CMIP6 downscaling (9~months)}

Can this model(s) be used to downscale cheaply from climate models?

\begin{itemize}
    \item Input: CMIP6 ensemble members sea surface wind vectors.
        For Models with output which is 
        similar enough to ECMWF for it to be plausible that
        it will generalise.
    \item Model: The model(s) that has been verified
        in the previous step.
    \item Output: Sea surface height (SSH) along the
        East US
        coastline maximum during each event storm
        event above a threshold in each extreme event.
\end{itemize}

From this output we could then:

\begin{enumerate}
 \item Fit a Generalised Pareto distribution of each ensemble 
       member (90 years of non-stationary data).
    
 \item However, given that this data is highly non-stationary,
    it might make more sense to fit the generalised extreme
    value distribution across all ensemble members for a given
    experiment over say each decade (c.~53 ensembles\footnote{
    I was able to easily extract 53 ensemble members on
    Pangeo~(\cite{Abernathey2017})
    for the MRes} x10 years $\approx$ 530 years of
    data  (although not identically
    distributed given different model biases)).
\end{enumerate}

For the second possibility, we could attempt to remove 
the different model biases from each ensemble member.
This would probably require looking at how biased they were in
the  historical period, and then assuming that some aspect of this is stationary.


\subsubsection*{Final output (if all this were to work well)}

\begin{itemize}
    \item A return level curve for each point along the
    East US coast,
    for each decade of the century, 
    for each CMIP6 experiment
    (SSP585 etc.).
\end{itemize}

Example headline: \textbf{``by reducing CO2 emissions 
by W\% the chance of a storm surge above Xm
 at Y in 2050-60 will reduce
by Z\% ($\pm$ A\%)''}\footnote{Where W is the difference between two 
SSP CMIP6 experiments.}


\subsection*{Writing up PhD (6 months)}

Writing up the thesis and tying up any loose ends. Push any publishable work through, and publicise any useful results or
code.

\printbibliography

\end{document}
