\begin{figure}
\centering
\includegraphics[width=1\linewidth]{../surge/plots/3d_plots/3d_plotnorlean.pdf}
 \caption{High pass filter (lower two panels) makes small difference.}
 \label{fig:tau-tau-r-no}
\end{figure}


\begin{figure*}
\centering
 \hspace{-40pt} \includegraphics[width=0.7\linewidth]{../surge/plots/rmlr.pdf}
  \vspace{-15pt}
 \caption{Huber regression generalises less well than MLR.}
 \label{fig:tau-tau-resp}
 \hspace{-40pt} \includegraphics[width=0.4\linewidth]{../surge/plots/reg_angle.png}
  \vspace{-15pt}
 \caption{\texttt{np.arctan2(c0, c1)}}
  \label{fig:tau-tau-angle}
  \hspace{-40pt} \includegraphics[width=0.6\linewidth]{../surge/plots/adj_reg_mag.pdf}
   \vspace{-15pt}
  \caption{($\bar{r^2}$)\texttt{*np.square(np.sqrt(c0) + np.sqrt(c1))}. }
   \label{fig:tau-tau-responsiveness}
\end{figure*}

Figure~\ref{fig:tau-tau-r-no} shows a single example of working out the linear
responsiveness (or linear predictability).

Figure~\ref{fig:tau-tau-resp} shows that when this same metric is applied along \texttt{eUS}
for each year in \texttt{tyr}.

Figure~\ref{fig:tau-tau-angle} shows that the angle is highly correlated with
the bearing of the coast that was defined in §~\ref{sec:convexity}.

Figure~\ref{fig:tau-tau-responsiveness} creates a metric for the
responsiveness.
