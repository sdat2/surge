\subsection{Moist thermodynamics}
\label{sec:moist-thermodynamics}

The law of thermodynamics for a PV system is,

\begin{equation}
d U=T d S-P d V,
\end{equation}

which can be adapted to the moist form, as in
Appendix 1 Emanuel 1986~\cite{emanuel1986air}.

\begin{equation}
T d s^{*}=d u+p d \alpha-L d q^{*}
\end{equation}

where $u$ isn the internal energy, $L$ is the heat of vaporisation,
and $q^{*}$ is the saturation mixing ration.
We also define the saturated  moist enthalpy,

\begin{equation}
h^{*} \equiv u+p \alpha-L q^{*},
\end{equation}

and can therefore show that,

\begin{equation}
d h^{*}=T d s^{*}+\alpha d p.
\end{equation}

This implies that,

\begin{equation}
\begin{array}{l}
\left(\partial h^{*} / \partial p\right)_{s^{*}}=\alpha, \\
\left(\partial h^{*} / \partial s^{*}\right)_{p}=T,
\end{array}
\end{equation}

As $q^{*}$ is only a function of temperature and pressure,
$h^{*}$ is a state variable. Therefore it is a function of
any of other two state variables. For $p$ and $s^{*}$,

\begin{equation}
\left(\frac{\partial}{\partial s^{*}}\right)_{p}\left(\frac{\partial h^{*}}{\partial p}\right)_{s^{*}}=\left(\frac{\partial}{\partial p}\right)_{s^{*}}\left(\frac{\partial h^{*}}{\partial s^{*}}\right)_{p},
\end{equation}

which implies,

\begin{equation}
\left(\frac{\partial \alpha}{\partial s^{*}}\right)_{p}=\left(\frac{\partial T}{\partial p}\right)_{s^{*}}.
\end{equation}

\paragraph{Tropical cyclone derivation}

From Section 3 Emanuel 1986~\cite{emanuel1986air}.
Air rushes inwards along the isotherm from $r_{0}$ acquiring
incremental latent heat

\begin{equation}
\Delta Q_{1}=\int_{\theta_{e a}}^{\theta_{e}} C_{p} T_{B} d \ln \theta_{e}=C_{p} T_{B} \ln \frac{\theta_{e}}{\theta_{e a}},
\end{equation}

Where $\theta_{e a}$ is the equivalent potential temperature at
$r_{0}$. The air ascends at constant entropy to the lower stratosphere
at a large radius. The air loses enough total heat through radiation cooling
to return to ambient temperature $\theta_{e}$ so

\begin{equation}
\Delta Q_{2}=\int_{\theta_{e}}^{\theta_{a a}} C_{p} T_{\mathrm{out}} d \ln \theta_{e}=-C_{p} \bar{T}_{\mathrm{out}} \ln \frac{\theta_{e}}{\theta_{e a}},
\end{equation}

where $\bar{T}_{\mathrm{out}}$ is given by,

\begin{equation}
\bar{T}_{\mathrm{out}} \equiv \frac{1}{\ln \theta_{e}^{*} / \theta_{e a}} \int_{\mathrm{ln} \theta_{e a}}^{\ln \theta_{e}} T_{\mathrm{out}} d \ln \theta_{e}^{*}.
\end{equation}

the total heating is therefore

\begin{equation}
\Delta Q=\Delta Q_{1}+\Delta Q_{2}=C_{p} T_{B} \epsilon \ln \frac{\theta_{e}}{\theta_{e a}},
\end{equation}

where $\epsilon$
$
\epsilon \equiv\left(T_{B}-s\bar{T}_{\text {out }}\right) / T_{B}
$
is the thermodynamic efficiency and $T_b$ is the temperature at the top of the
boundary layer. By the first law of thermodynamics, the total heating
must balance the work done,

\begin{equation}
\Delta Q=W_{P B L}+W_{0}
\end{equation}

where $W_{PBL}$ and $W_0$ are the work done in the boundary layer and outflow
respectively. $W_0$ is proportional to the kinetic energy required to
bring the angular momentum of the outflow back to the ambient value:

\begin{equation}
\begin{aligned}
W_{0} &=\frac{1}{2} \Delta V^{2}=\frac{1}{2}\left[\left(\frac{M}{r_{1}}
-\frac{1}{2} f r_{1}\right)^{2}-\left(\frac{M_{0}}{r_{1}}-\frac{1}{2} f r_{1}\right)^{2}\right] \\
&=\frac{1}{2}\left[\frac{M^{2}-M_{0}^{2}}{r_{1}^{2}}+f\left(M_{0}-M\right)\right]
\end{aligned}
\end{equation}

where we have used,

\begin{equation}
M \equiv r V+\frac{1}{2} f r^{2}
\end{equation}

where $r_1$ is the large radius at which the exchange place, so that,

\begin{equation}
\lim _{r_{1} \rightarrow \infty} W_{0}=\frac{1}{2} f\left(M_{0}-M\right)=\frac{1}{4} f^{2}\left(r_{0}^{2}-r^{2}\right)-\frac{1}{2} f r V.
\end{equation}

Which can be then used to infer the work done in the boundary layer:

\begin{equation}
W_{\mathrm{PBL}}=C_{p} T_{B} \epsilon \ln \frac{\theta_{e}}{\theta_{e a}}+\frac{1}{2} f r V-\frac{1}{4} f^{2}\left(r_{0}^{2}-r^{2}\right)
\end{equation}


we can then use Bernoulli's equation,

\begin{equation}
p+\frac{1}{2} \rho V^{2}+\rho g h=\text { constant },
\end{equation}

and the Exner function is used in place of pressure:

\begin{equation}
\pi=\left(p / p_{0}\right)^{R / C_{p}}.
\end{equation}

where $C_p$ is the heat capacity at constant pressure,
$p_{0}$ is the ambient pressure, to see that,

\begin{equation}
\frac{1}{2}
V^{2}+C_{p} T_{B} \ln \pi+W_{\mathrm{PBL}}=0 \quad \text { at } \quad z=0.
\end{equation}


Using the gradient wind equation,

\begin{equation}
\alpha \frac{\partial p}{\partial r}=\frac{V^{2}}{r}+f V=\frac{M^{2}}{r^{3}}-\frac{1}{4} f^{2} r
\end{equation}

an expression for the pressure is then,

\begin{equation}
\begin{array}{r}
\ln \pi+\frac{1}{2} r \frac{\partial \ln \pi}{\partial r}+\epsilon \ln \frac{\theta_{e}}{\theta_{e a}}-\frac{1}{4} \frac{f^{2}}{C_{p} T_{B}}\left(r_{0}^{2}-r^{2}\right)=0 \\
\text { at } z=0
\end{array}
\end{equation}

There is a further correction term decreasing the central pressure,
created by removing angular momentum and heat at finite $r_1$
rather than the limit as $r_1\to\infty$:

\begin{equation}
\delta \ln \pi=\frac{-\frac{1}{8} \frac{f^{2} r_{0}^{4}}{r_{1}^{2}}}{C_{p} T_{B}\left[1-\epsilon\left(1+\frac{L q_{a}^{*} \mathrm{RH}_{s}}{R T_{s}}\right)\right]},
\end{equation}

which is insignificant when $r_1=r_0$, and only important if $r_0$ is very large.



\begin{table}[h!]
\centering
\begin{tabular}{lll}
\hline \hline
\textbf{Quantity} & \textbf{Scale} & \textbf{Typically c.}\\
\hline
Length & $\chi_s^{1/2}f^{-1}$ & 1000 km \\
Time & $C_D^{-1}H\chi_{s}^{-1/2}$ & 16 hr \\
Azimuthal velocity & $\chi_s^{1/2}$ & 60 m s$^{-1}$\\
Radial velocity & $\frac{1}{2}C_d\chi_s f^{-1}H^{-1}$ & 10 m s$-1$\\
Vertical velocity & $C_D \chi_s^{1/2}$ & 6 cm s$-1$\\
\hline \hline
\end{tabular}\\
\textit{Table 1 from~\cite{emanuel1991theory}.}
\caption{The different scales within a tropical cyclone (TC).
$\chi_s\equiv(T_s-T_t)(s_{0}^{*}-s_a)$, the thermodynamic disequilibrium parameter,
 where $T_s$ is the temperature
of the ocean,$T_t$ is the ambient temperature of the tropopause,
$s_{0}^{*}$ is the saturation entropy of the ocean surface and
$s_a$ is the entropy of the normal tropical atmosphere near sea level.
$s_{0}^{*}-s_a$ represents the thermodynamic disequilibrium of the ocean-atmosphere
system. $H\equiv$ depth of convecting layer, $f\equiv$ twice the
local vertical component of Earth's angular velocity, $C_D\equiv$
dimensionless exchange coefficient.}

\label{tab:hurr-scales}
\end{table}

