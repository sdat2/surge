
\begin{table}[h!]
\centering
\begin{tabular}{lll}
\hline \hline
\textbf{Quantity} & \textbf{Scale} & \textbf{Typically c.}\\
\hline \\
Length & $\chi_s^{1/2}f^{-1}$ & 1000 km \\ \\
Time & $C_D^{-1}H\chi_{s}^{-1/2}$ & 16 hr \\ \\
Azimuthal velocity & $\chi_s^{1/2}$ & 60 m s$^{-1}$\\ \\
Radial velocity & $\frac{1}{2}C_d\chi_s f^{-1}H^{-1}$ & 10 m s$^{-1}$\\ \\
Vertical velocity & $C_D \chi_s^{1/2}$ & 6 cm s$^{-1}$\\
\hline \hline
\end{tabular}\\
\textit{Table 1 from~\cite{emanuel1991theory}.}
\caption{The different scales within a tropical cyclone (TC).
$\chi_s\equiv(T_s-T_t)(s_{0}^{*}-s_a)$, the thermodynamic disequilibrium parameter,
 where $T_s$ is the temperature
of the ocean,$T_t$ is the ambient temperature of the tropopause,
$s_{0}^{*}$ is the saturation entropy of the ocean surface and
$s_a$ is the entropy of the normal tropical atmosphere near sea level.
$s_{0}^{*}-s_a$ represents the thermodynamic disequilibrium of the ocean-atmosphere
system. $H\equiv$ depth of convecting layer, $f\equiv$ twice the
local vertical component of Earth's angular velocity, $C_D\equiv$
dimensionless exchange coefficient.}

\label{tab:hurr-scales}
\end{table}
