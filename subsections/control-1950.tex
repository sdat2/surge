\subsubsection{\texttt{control-1950} daily means}

\label{sec:control-1950-intro}

The second source of data was the \texttt{control-1950} HADGEM3 experiment (\texttt{c50}),
made available via the PRIMAVERA project.
The HADGEM3 coupled climate model~\cite{williams2018met, FurtherInfo},
 was the Met Office's submission to CMIP6,
although \texttt{control-1950} is not a standard experiment in CMIP6~\cite{eyring2016overview}
.\footnote{CMIP is the coupled model intercomparison program,
whose models form an important part of the scientific basis for the IPCC
reports (e.g.~\cite{SROCC}).}

\texttt{control-1950} has its forcing kept at
1950s levels for 101 360-day years. Outputs are recorded as daily
means~\cite{williams2018met, FurtherInfo}.\footnote{\url{https://view.es-doc.org/
        ?renderMethod=name&project=cmip6&type=cim.2.designing.NumericalExperiment&client
        =esdoc-url-rewrite&name=control-1950}}.
Due to time constraints, only the SSH outputs were used in this report.

\paragraph{TC deficit} As is shown in Figure 19 of~\cite{williams2018met},
 HADGEM3 seems to systematically underrepresent
 the number of TCs compared to observations,
 particularly in the North Atlantic.
 This is probably caused by the poor resolution
 of cyclogenesis~\cite{tomassini2017interaction},
 but could also be exacerbated by the tropical
 cold tongue bias in all CMIP models~\cite{camargo2013global} (see Seager et~al.~2019~\cite{seager2019strengthening})
 which is yet to be rectified.
