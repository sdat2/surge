\section{Conclusion}
\label{sec:8_Conclusion}

Storm surges rise out of the sea due to the stress applied to the water by
the atmosphere, with their size dependent to the size of the stress above and
the slope of the bathmetry beneath. This study might provide some basis for
future studies of the coastline, as it radically reduces the computational cost
of training algorithms, by choosing only those areas that are needed (the
land adjacent cells).

As set out in Figure~\ref{fig:return_hyp} combining the insights of
extreme value and hurricane potential intensity theory will allow
us to more accurately estimate the hazards posed to coastal communities
in the future.

As shown in our first order thermodynamic understanding, it seems likely that
global warming will lead to an increase in hurricane intensity, independent
of the imperfections of the current round of global climate models,
but instead rests on simple physical argument.
Even if the most severe outcomes may not occur,
it would be as inprudent to not take account of them as it would be
to not buy fire insurance for your house.
