\section{Coastlines within a climate model}
\label{sec:coast}
\subsection{Selection}
 \subsubsection{Fractal coastline}
 \label{sec:fractal}
 The coastline appears to be a self-similar fractal~\cite{mandelbrot1967long,
 richardson1961problem}, by which is meant that it has the same features at all
 scales, so that the length of the coast increases without limit as resolution
 is increased.
 A consequence of this is that a coastline has no defined length without
 reference to the resolution at which it is measured.
 This presents some problem when comparing different climate models as they will have different
 numbers of coastal cells. It has been observed that the fractal dimension changes
  along the East US coast~\cite{jiang1998fractal}
 and this would mean that the relative number of coastal cells between different
 points will also change as the resolution is increased.

\subsubsection{Algorithms}
\label{sec:coast-algorithms}
Algorithms~\ref{algo:Selection}~\&~\ref{algo:Sort} first extract the coastline
from the surface slice of the ocean field and then pick out a single chain of
coastal points between two chosen ends. The list of points can be further
manually sorted to remove unwanted regions such as large bays (e.g.~Chesapeake~bay)
and the occasional river island (e.g.~Shanghai bay's island). Once this is
done \texttt{xarray} can be used to quickly extract these points
from the climate model to make a new NetCDF file.



\begin{algorithm}[H]
\dontprintsemicolon
% \DontPrintSemicolon % Some LaTeX compilers require you to use \dontprintsemicolon instead
\caption{Coastal Selection}
\label{algo:Selection}
\KwIn{Grid: boolean grid (True for NaN values (land))}
\For{indices in Grid}{
  \If {Grid[indices] == False}{
     \If{ any of Grid[index\_x $\pm$ 1, index\_y $\pm$ 1] are True}
        {\textbf{append} indices to List\;}
     }
}
\KwOut{List: list of indices}
\end{algorithm}

\begin{algorithm}[H]
\dontprintsemicolon
% \DontPrintSemicolon % Some LaTeX compilers require you to use \dontprintsemicolon instead
\caption{Coastal Sort}
\label{algo:Sort}
\KwIn{Input\_List: Slimmed output of Algorithm~\ref{algo:Selection}}
\KwIn{First\_Index\_Pair: Point at one end of the coast }
\KwIn{Final\_Index\_Pair: Point at other end of the coast }

\texttt{tmp} $\gets$ First\_Index\_Pair\;

\Repeat{\texttt{tmp} = Final\_Index\_Pair}{
 \textbf{append} \texttt{tmp} to Output\_List\;
 \textbf{remove} \texttt{tmp} from Input\_List\;
 \texttt{tmp} $\gets$ Input\_List[\textbf{min}$_{k}$(\textbf{distance}(\texttt{tmp}, Input\_List[k]))]\;
}
\textbf{append} Final\_Index\_Pair to Output\_List\;

\KwOut{Output\_List: sorted list of indices}
\end{algorithm}


\subsubsection{Sample coastlines}
The first coastline chosen was the eastern coastline of the mainland
United States, so that it could be compared with~\cite{ZannaPreprint},
and because the topography of the NEMO models is most accurate along the
US coastline.

Secondly I extracted the coastline from Hue, Northern Vietnam, to China's
border with North Korea, removing Shanghai bay island.

Finally we extracted the eastern coastline of Honshu island, Japan.

\subsection{Metrics}
\subsubsection{Motivation}
A quick look at the coast revealed that certain features stuck out:
high kurtosis at New Orleans (NO), at the end of a shallowly sloping bay;
low kurtosis at Miami (MM), at the end of the headland of Florida.
But it would be interesting to find out what is really the controlling
factor, or whether either of these are truly predictive of storm surge risk.

To assess this question, it is necessary to quantify these two coastal attributes
and given that the coast is fractal (§~\ref{sec:fractal}), these metrics must also be defined at
all scales.

\subsubsection{Convexity}
Given a selected coastline, we have a list of points of latitude and longitude.
We can define the normal bearing to the coast.

This provides a convexity value for every smoothing length $\sigma$.

\subsubsection{Bathymetric gradient}
The simplest way of working this out is to look at the
bathymetry netCDF file that the model was run with (ETOPO2v2-derived in this
 case (§~\ref{sec:nemo})), and work out where some set of chosen isobaths are.
Once these isobaths have been selected, the cartesian distance
within the model grid from every point on
the coast to that isobath are calculable.

This provides a distance-to-isobath value for each of the isobaths
queried.
