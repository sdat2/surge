\section{Discussion}
\label{sec:7_Discussion}

\subsection{Future work}
\label{sec:future}

Different models have different systematic biases
and also different resolutions. It would be interesting to see how
the responsiveness changes as the resolution decreases, as the difference
between the value at 4km distance (c.~1.6m) of Hurricane Katrina (see Figure~\ref{fig:katrina}),
is smaller than the value observed at the coast itself (c.~9m according to news reports).
It must increase with a bound.
Combining the climate models outputs with a SLOSH model for downscaling could produce
an output which could more directly be compared against the tidal gauges
without the bias created by measuring SSH from the centre of the coastal
cell.

Models like LLC4320 include tides (despite great computational expense), and these tides introduce
non-linear tide-surge interaction~\cite{feng2019characteristics, arns2020non}.
Dangerous storm surges, such as the 1938 New England
Hurricane, coincided with spring high tides,
 and according to Arns~et~al.~April~2020~\cite{arns2020non}
these effects are strongest on the UK's East Coast,
the US east coast, and southern Japan, i.e. most of the coastlines in
this report.

Data is available from tidal gauges along the U.S.~coastline and worldwide as
in \cite{tadesse2020data, arns2020non}.
Frequencies of data are variable, and the API to extract this data
is difficult to implement.


There are insufficient TCs in CMIP models~\cite{camargo2013global},
because the processes involved
in cyclogenesis are hard to resolve in climate models.
As is shown in Figure 19 of~\cite{williams2018met},
 for HadGEM3 this is particularly a problem in the,
 particularly in the North Atlantic §~\ref{sec:cyclogenesis}.
CMIP models are colder in the tropics than observations~\cite{camargo2013global}.
For a recently published explanation see Seager et al.~2019~\cite{seager2019strengthening}.
This may lead to the lack of TCs relative to observations~\cite{tomassini2017interaction}.

The theory of PI is well supported, but a few super-intense tropical cyclones have been observed,~\cite{camargo2019tropical}
which could be a result treating the bulk parameters as constant in the theory (§~\ref{sec:param}).

The sea surface does reach its steady state as soon as sea surface
stress is applied, which this leads
to the regression being quite distorted, as the assumption would be that data
is in steady state, when it actually depends significantly on the previous points.
It might therefore make sense to give whichever ML algorithm
the series of previous $\eta$ values for some number of lags.


\subsection{Summary of systematic errors}
\label{sec:sys-errors}
Table~\ref{tab:syst_error} shows the list of systematic errors that effect the
observed hazard, most of which will lead to an underestimate.

\begin{table}[ht]
\resizebox{\columnwidth}{!}{%
\begin{tabular}{l L{8.5cm}}
%\hline \hline
\textbf{Effect} & \textbf{Source}\\
\hline
($-$) & Coarsening of time to daily outputs.\\
($-$) & Lack of cyclogenesis  $\implies$ not enough
        TCs~\cite{tomassini2017interaction, williams2018met, FurtherInfo}.\\
($\pm$) & Parameterisation of heat transfer $\implies$ wrong TC strength.\\
($\pm$) & Parameterisation of wind stress $\implies$ wrong TC surge.\\
($-$) & SSH 4km out from the coast (centre of cell).\\
($\pm$) & No non-linear surge-tide interaction.\\
($-$) & Limited wave build up.\\
($-$) & Long TC return periods $> 100$~yrs (e.g.~New England).\\
\hline
\end{tabular}
}
\caption{Systematic sources of error for the estimate of TC-induced storm surge frequency
and severity.}
\label{tab:syst_error}
\end{table}

