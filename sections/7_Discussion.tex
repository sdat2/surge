\section{Discussion}
\label{sec:7_Discussion}

\subsection{Future work}
\label{sec:future}
\subsubsection{Model intercomparison}
Different models have different systematic biases (in cyclogenesis etc.)
and also different resolutions. It would be interesting to see how
the responsiveness changes as the resolution decreases, as the difference
between the value at 4km distance (~2m) of the Katrina Hurricane (see Figure~\ref{fig:katrina}),
is quite a lot smaller than the actual value observed at the coast itself (c. 9m according to news reports).
Presumably it must increase with a bound. As model resolution increases the model
timesteps have to increase in frequency so as to avoid numerical blow-up.

\subsubsection{Gauges}
Data is available from tidal gauges along the U.S.~coastline and worldwide as
in \cite{tadesse2020data, arns2020non}.
Frequencies of data are variable, and the API to extract this data
is difficult to implement.

\subsubsection{Tides}
Models like LLC4320 include tides, and these tides introduce
non-linear tide-surge interaction~\cite{feng2019characteristics, arns2020non} which should be studied.
Famously dangerous storm surges, such as the 1938 New England
Hurricane, coincided with spring high tides, and according to Arns~et~al.~April~2020~\cite{arns2020non}
these effects are strongest on the UK's East Coast,
the US east coast, and southern Japan, i.e. most of the coastlines in
this report.

\subsubsection{Inundation}
Combining the climate models outputs with a SLOSH model could produce
an output which could more directly be compared against the tidal gauges
without the bias created by measuring SSH from the centre of the coastal
cell.


\subsection{Cyclone Problem}
There are insufficient TCs in climate models, because the processes involved
in cyclogenesis are hard to resolve in climate models. This may be
exacerbated by the cold bias (see Seager et~al.~2019~\cite{seager2019strengthening})

A few super-intense tropical cyclones have been observed, however the community
 trusts the assumptions made in the
Carnot cyclone model~\cite{camargo2019tropical}.
One deficiency is treating the bulk parameters
as constant (§~\ref{sec:param}).

\subsection{Sea surface Hysteresis}
The sea surface does not instantly reach its steady state when the sea surface
stress is applied, and it does not recede from its high level, this leads
to the regression being quite distorted, as the assumption would be that data
is iid, when it actually depends significantly on the previous points.
It might therefore make sense to give whichever ML algorithm
the series of previous $\eta$ values for some number of lags.


\subsection{Summary of Systematic Errors}
\label{sec:sys-errors}
Table~\ref{tab:syst_error} shows the list of systematic errors that effect the
answer.

\begin{table}[ht]
\resizebox{\columnwidth}{!}{%
\begin{tabular}{l L{8.5cm}}
%\hline \hline
\textbf{Effect} & \textbf{Source}\\
\hline
($-$) & Coarsening of time to daily outputs.\\
($-$) & Lack of cyclogenesis  $\implies$ not enough
        TCs~\cite{tomassini2017interaction, williams2018met, FurtherInfo}.\\
($\pm$) & Parameterisation of heat transfer $\implies$ wrong TC strength.\\
($\pm$) & Parameterisation of wind stress $\implies$ wrong TC surge.\\
($-$) & SSH 4km out from the coast (centre of cell).\\
($-$) & No non-linear surge-tide interaction.\\
($-$) & Limited wave build up.\\
($-$) & Long TC return periods $> 100$~yrs (e.g.~New England).\\
($\pm$) & Inaccurate bathymetry.\\
\hline
\end{tabular}
}
\caption{Systematic sources of error for the estimate of TC-induced storm surge
hazard.}
\label{tab:syst_error}
\end{table}

