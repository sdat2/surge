\begin{table}[ht]
\resizebox{\columnwidth}{!}{%
\begin{tabular}{l L{8.5cm}}
\hline \hline
\textbf{Effect} & \textbf{Source}\\
\hline
($-$) & Coarsening of time to daily outputs.\\
($-$) & Lack of cyclogenesis~\cite{tomassini2017interaction, williams2018met, FurtherInfo}.\\
($\pm$) & Parameterisation of heat transfer $\implies$ wrong TC strength.\\
($\pm$) & Parameterisation of wind stress $\implies$ wrong TC surge.\\
($-$) & SSH 4km out from the coast (centre of cell).\\
($-$) & Lacking non-linear surge-tide interaction.\\
($-$) & Limited wave build up.\\
($-$) & Long TC return periods $> 100$~yrs (e.g.~New England)
mean that \texttt{c50} is insufficient to capture distribution.\\
($\pm$) & Inaccurate bathymetry.\\
($-$) & \texttt{c50} records daily means, but peak value occurs over c.~an hour
(see Figure~\ref{fig:katrina}).\\
($+$) & If GEV were fitted naively when there is a PI limit.\\
\hline \hline
\end{tabular}
}
\caption{Systematic sources of error for the estimate of TC-induced storm surge
hazard, where the sign is the effect that this error would have had on our estimate
of hazard relative to the true hazard.}
\label{tab:syst_error}
\end{table}
