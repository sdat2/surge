\section{Supplementary results}
\label{sec:sup-res.}

\subsection{Summary variable statistics}
\label{sec:sum-var-stat}

\paragraph{Gaussian processes (GPs)}
\begin{itemize}
\item Equations 2.13-2.14 in GPML~\cite{williams2006gaussian}.
\begin{align}
m(\mathbf{x})&=&\mathbb{E}[f(\mathbf{x})] % \tag{Mean}
\\
k\left(\mathbf{x}, \mathbf{x}^{\prime}\right)&=&\mathbb{E}
\left[(f(\mathbf{x})-m(\mathbf{x}))\left(f\left(\mathbf{x}^{\prime}\right)
-m\left(\mathbf{x}^{\prime}\right)\right)\right]
%\tag{Covariance}
\\
f(\mathbf{x})& \sim& \mathcal{G} \mathcal{P}\left(m(\mathbf{x}),
 k\left(\mathbf{x}, \mathbf{x}^{\prime}\right)\right)%\tag{GP}
\end{align}
\item Can assume $m(\mathbf{x})=0$ without terrible consequences.
 \item Thought should be put into the form of $k\left(\mathbf{x},
  \mathbf{x}^{\prime}\right)$~\cite{duvenaud2014automatic}.
\item GPs assume that there is a Gaussian error around each point,
      but this is often not the case in real variables.
      However, it is often possible to transform to
      a space where this is the case, Krige there,
      and then transform
      back~\cite{snelson2004warped}.\footnote{\url{http://mlg.eng.cam.ac.uk/zoubin/papers/gpwarp.pdf}}
      \includegraphics[width=\linewidth]{images/example-images/warped-example.png}\\
      \textit{Figure 1 from~\cite{snelson2004warped}}
\item Lewis Fry Richardson (1948) showed that the estimates
      of deaths during a war had
      symmetric error bars in logarithmic space~\cite{richardson1948variation}.
      \cite{snelson2004warped}~mentions this as a common trick.
 \end{itemize}

\begin{figure*}[htb!]
    \centering
    \includegraphics[width=0.85\linewidth]{../surge/plots/stats_points_plot_2.pdf}
    \vspace{-7pt}
    \caption{An attempt to check whether there are discontinuities
     over the special points, to explain the offsets of the means.
     The points are ordered from Eastport (EP) (furthest north east),
     to Rockport (RP) (furthest south west). From the plot it is visible
     that there is a low frequency annual component to each signal, and that
     the excursions are more common during autumn and winter, with
     relative calm in spring.}
    \label{fig:individual_zos}
    \includegraphics[width=0.85\linewidth]{../surge/plots/ahh/ahhhhh4.pdf}
    \vspace{-7pt}
    \caption{Mean SSH over points of the US coast over the two year period. There is a strong yearly periodicity in $\eta$ (and its variance?).
     Kriged with a Sobol quasirandom subsample~\cite{sobol1967distribution} of 6000 time points.
     }
    \label{fig:gauss-mean}
\end{figure*}

