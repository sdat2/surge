\begin{figure}[htb!]
\centering
\includegraphics[width=\linewidth]{../surge/plots/point_choosing/standard_loc_test.pdf}

\vspace{-15pt}

\caption{The east coast of the United States (\texttt{eUS}), points chosen to be similar to
Y20~\cite{ZannaPreprint}.}
\label{fig:america}


\includegraphics[width=\linewidth]{../surge/plots/vin-china-choice.pdf}

\vspace{-11pt}

\caption{The northern coast of Vietnam
and mainland coastline of China (\texttt{vc}),
with the coastal cells chosen shown in green,
and the depths from ETOPO2v2.}
\label{fig:vin-china}
\end{figure}


\section{Coastlines within a climate model}
\label{sec:coast}
\subsection{Selection}
 \subsubsection{Fractal coastline}
 \label{sec:fractal}
 The coastline appears to be a self-similar fractal~\cite{mandelbrot1967long,
 richardson1961problem}. This means that it has the same features at all
 scales, so that the length of the coast increases without limit as resolution $G$
 is increased,
 \begin{equation}
 L(G)=M G^{1-D},
 \end{equation}
 where $L$ is the total length, $G$ is the length of a side,
 and $D>1$ is the fractal dimension~\cite{mandelbrot1967long}.
 A consequence of this is that a coastline has no defined length without
 reference to the resolution at which it is measured.
 This will present problems when comparing
 climate models with different resolutions.
 The fractal dimension changes
  along the East US coast ($1.11<D<1.20$ in one degree latitude bins)~\cite{jiang1998fractal}
 and this would mean that the relative number of coastal cells between different
 points will change as the resolution is increased.
 Therefore locations are labelled as listed in
 Figures~\ref{fig:america}~\&~\ref{fig:vin-china}.

\subsubsection{Algorithms}
\label{sec:coast-algorithms}
Algorithms~\ref{algo:Selection}~\&~\ref{algo:Sort} first extract the coastline
from the surface slice of the ocean field and then pick out a single chain of
coastal points between two chosen ends. The list of points can be further
sorted to remove unwanted regions.
\texttt{xarray}~\cite{hoyer2017xarray} is then used to quickly extract
a single NetCDF file for these points from a series of large files.



\begin{algorithm}[H]
\dontprintsemicolon
% \DontPrintSemicolon % Some LaTeX compilers require you to use \dontprintsemicolon instead
\caption{Coastal Selection}
\label{algo:Selection}
\KwIn{Grid: boolean grid (True for NaN values (land))}
\For{indices in Grid}{
  \If {Grid[indices] == False}{
     \If{ any of Grid[index\_x $\pm$ 1, index\_y $\pm$ 1] are True}
        {\textbf{append} indices to List\;}
     }
}
\KwOut{List: list of indices}
\end{algorithm}

\begin{algorithm}[H]
\dontprintsemicolon
% \DontPrintSemicolon % Some LaTeX compilers require you to use \dontprintsemicolon instead
\caption{Coastal Sort}
\label{algo:Sort}
\KwIn{Input\_List: Slimmed output of Algorithm~\ref{algo:Selection}}
\KwIn{First\_Index\_Pair: Point at one end of the coast }
\KwIn{Final\_Index\_Pair: Point at other end of the coast }

\texttt{tmp} $\gets$ First\_Index\_Pair\;

\Repeat{\texttt{tmp} = Final\_Index\_Pair}{
 \textbf{append} \texttt{tmp} to Output\_List\;
 \textbf{remove} \texttt{tmp} from Input\_List\;
 \texttt{tmp} $\gets$ Input\_List[\textbf{min}$_{k}$(\textbf{distance}(\texttt{tmp}, Input\_List[k]))]\;
}
\textbf{append} Final\_Index\_Pair to Output\_List\;

\KwOut{Output\_List: sorted list of indices}
\end{algorithm}


\subsubsection{Sample coastlines}
\label{sec:coast-sample}
The primary coastline extracted was the eastern mainland US coast (\texttt{eUS}),\footnote{
Removing Chesapeake Bay, Long Island Sound, and Pamlico Sound.
}
so it could be compared to Y20~\cite{ZannaPreprint},
and because ETOPO2 topography is most accurate along the
US coast (see Figure~\ref{fig:america}).
I extracted the coastline from Hue, Vietnam, to China's
border with North Korea (\texttt{vc}),\footnote{
Removing Shanghai bay island.} (see Figure~\ref{fig:vin-china}).
Finally we extracted the eastern coastline
of Honshu island, Japan (\texttt{hon}).



\subsection{Metrics}
\subsubsection{Motivation}
Figure~\ref{fig:ssh_stats_america} shows
high kurtosis at New Orleans (NO), at the end of a shallowly sloping bay;
low kurtosis at Miami (MM), at the end of the headland of Florida.

To assess what controls this,
it is necessary to quantify these two coastal attributes,
and given that the coast is fractal (§~\ref{sec:fractal}),
these metrics must also be defined at all scales.

\subsubsection{Convexity}
\label{sec:convexity}

For each coastline,
there is a list of latitude and longitude points.
We can define the normal bearing to the coast,

\begin{equation}
B_i^{\prime}=\operatorname{arctan} 2
\left(\phi_{i+1}-\phi_{i-1},\; \chi_{i+1}-\chi_{i-1}\right),
\label{eq:bearing}
\end{equation}

where $\phi_{i}$ is latitude, and $\chi_{i}$ is the longitude of the coastal points.
This can be smoothed,

\begin{equation}
B_i=\operatorname{arctan} 2
\left(S_i,\; C_i\right)
\label{eq:bearing}
\end{equation}

with,

\begin{equation}
S_i = \sum_{j=i-4\sigma}^{j=i+4\sigma} \mathcal{N}(i, \sigma^{2})\cdot \sin{( B_{j}^{\prime})},
\end{equation}
\begin{equation}
C_i  =  \sum_{j=i-4\sigma}^{j=i+4\sigma} \mathcal{N}(i, \sigma^{2})\cdot  \cos{( B_{j}^{\prime})},
\end{equation}

with reflective boundary conditions.

An example of this is shown in Figure~\ref{fig:angle_heatmap}} for the\texttt{eUS}.
We define the symmetric derivative of this smoothed angle,

\begin{equation}
\frac{d B_i}{d \mathrm{pt}}
\equiv \frac{B_{i+1}-B_{i-1}}{2},
\label{eq:bearing_derivative}
\end{equation}

where pt stands for the points along the coast.

This provides a convexity value for every smoothing length $\sigma$,\footnote{
To overcome the wrapping of the bearing at $2\pi$,
there is an additional step to add and subtract $2\pi$ from the
answer and choose the result with the smallest magnitude.
} as shown in Figure~\ref{fig:derivative}A-C. The reflective boundary conditions
ensure that the bearing derivative is 0 at either end of the coastline.





\begin{figure}[htb!]
\centering
\includegraphics[width=\linewidth]{../surge/plots/angle_heatmap.pdf}
\caption{Normal bearing, $B$, along the east U.S.~coast.
         The Gulf of Mexico is visible as the curve between Rockport (RP)
         and Miami~(MM).
         }
\label{fig:angle_heatmap}

\includegraphics[width=\linewidth]{../surge/plots/derivative_heatmap.pdf}
\caption{Convexity metric along the east U.S.~coast.
         The three panels above show bay like concavity in blue, and convexity
         in red. At the largest $\sigma$ only the largest headlands such as
         Florida are visible (C), whereas the smallest bays are visible at a lower
        $\sigma$ (A).
}
\label{fig:derivative}
\end{figure}


\begin{figure}[htb!]
\includegraphics[width=\linewidth]{../surge/plots/bath_list.pdf}
\vspace{-25pt}

\caption{Isobaths plotted for the East US~coast.}
\label{fig:bath}
\includegraphics[width=\linewidth]{../surge/plots/distance_isobath.pdf}
\vspace{-25pt}

\caption{Distance to isobaths from points on East US~coast. MM is close to 500m
contour, as the bathymetry rapidly drops off, whereas at NO the drop off is more
gradual.}
\label{fig:isobath}
\includegraphics[width=\linewidth]{../surge/plots/isobath_correlate.pdf}
\vspace{-25pt}

\caption{The correlation matrix between the different isobath's distance's to
the East~US coast.}
\label{fig:isobath}
\end{figure}


\subsubsection{Bathymetric gradient}
\label{sec:bath-grad}
The simplest way of working this out is to look at the
bathymetry netCDF file that the model was run with (ETOPO2v2-derived in this
 case (§~\ref{sec:nemo})), and define the isobath as
 points that are both level or below a given depth,
  and adjacent to a point above that depth (Figure~\ref{fig:bath}).\footnote{This was accelerated by 3 orders of magnitude by \texttt{numba}~\cite{lam2015numba}.}
Once these isobaths have been selected, the distance
within the model grid from every point on
the coast to that isobath are calculated (Figure~\ref{fig:isobath}).

\FloatBarrier
