\documentclass[../main.tex]{subfiles}
\begin{document}
\section{Acknowledgements}
\subsection{Personal Thanks}
Dr.\ Dan Jones (British Antarctic Survey) has been an invaluable help throughout this project,
help in this project.
Dr.\ Laure Zanna (New York University) was a useful counsel as to what is feasible,
 and what the common views in the literature were.
Dr.\ Pierre Mathiot (Met Office) also deserves my thanks for
producing the high-resolution ORCA12 data that I used, and
for answering my queries.
I also received advice from Dr.\ Rory Bingham (Bristol).
I hope that in the fullness of time these collaborators will be rewarded
as co-authors, as we expand and publish the
work begun here.


\subsection{Resources Used}

This report was typeset in \LaTeX\
with \texttt{python3.matplotlib}~\cite{Hunter:2007} and \texttt{draw.io}~\cite{DrawIO}
for original figures, WebPlotDigitiser for data extraction~\cite{WebPlotDigitiser},
and Mathpix for maths extraction~\cite{mathpix}.
It makes extensive uses of the sci-kit learn library~\cite{scikit-learn} as highlighted in the report.
The \texttt{cmocean} package~\cite{thyng2016true}\footnote{\href{https://matplotlib.org/cmocean/}{https://matplotlib.org/cmocean/}}
provided oceanography standard colour maps.
I used Panolpy and \texttt{ncview}\footnote{\url{http://meteora.ucsd.edu/~pierce/ncview_home_page.html}}
for inspecting NetCDF files.

I made use of these textbooks:~\cite{roisin2010GFD,williams2011ocean,ITILA,
sivia2006data,landau1959course,faber1995fluid,williams2006gaussian,
murphy2012machine}.\footnote{\href{http://www.gaussianprocess.org/gpml/chapters/RW.pdf
}{http://www.gaussianprocess.org/gpml/chapters/RW.pdf}}
The report was written as instructed in the 1B Experimental Methods course
2016~\cite{presentation} and the Cavendish Guide to Writing Formal
Reports~\cite{LabNotes}, with writing style informed
by the Stanford Online Science Writing Course~\cite{ScienceWriting}
and McIntyre 1997~\cite{mcintyre1997lucidity}.
I found Stephen Pinker's book~\cite{pinker2015sense} especially useful and entertaining.

The code listing for this report is available as a \texttt{python3} repository on Gitlab~\cite{gitlab}.
I have tried to write using PEP8 style\footnote{\href{https://www.python.org/dev/peps/pep-0008/}{https://www.python.org/dev/peps/pep-0008/}}
and have been helped by \texttt{pylint} and Pycharm.\footnote{\href{https://www.jetbrains.com/pycharm/download/}{https://www.jetbrains.com/pycharm/download/}}
My use of \texttt{python3} is informed by the course given by Dr.~Buscher that I attended~\cite{SciCompNotes}
and by Bader (2017)~\cite{bader2017python}.
Errors are dealt with as per the Physics formula booklet~\cite{MathsFormulaBooklet}
by the \texttt{python3.uncertainties}
package~\cite{lebigot2010uncertainties}.\footnote{\href{https://pythonhosted.org/uncertainties/}{https://pythonhosted.org/uncertainties/}}

 To move and process the large quantity of climate data on the HPCs I used \texttt{tmux}
 to run \texttt{bash}, \texttt{python3}, and \texttt{fortran} scripts.

\end{document}
