\section{Discussion}
\label{sec:7_Discussion}

\label{sec:future}

Different models have different systematic biases
and resolutions. We could measure how
the responsiveness changes as the resolution decreases. The difference
between the value at 4km distance (c.~1.6m) of Hurricane Katrina (see Figure~\ref{fig:katrina}),
is smaller than the value observed at the coast itself (c.~9m in places).
It must increase with a bound.
Combining the coastline outputs with a high resolution indundation model
would allow direct comparison with tidal gauges.
Data is available from tidal gauges worldwide~\cite{tadesse2020data, arns2020non},
and particularly for the US.\footnote{
However frequencies of the data are variable,
and the data extraction API is poorly designed.}

Models such as LLC4320~\cite{Abernathey2017}  include tides,
 despite computational expense,
introducing non-linear tide-surge interactions~\cite{
feng2019characteristics, arns2020non}.
Dangerous storm surges, e.g.~the 1938 New England
Hurricane, coincided with spring high tides.
As harm is a convex function T20~\cite{taleb2019statistical, Taleb2012AntifragileH},
this interaction will increase the hazard of storm surges.
According to Arns~et~al.~2020~\cite{arns2020non}
these effects are strongest on the E~UK coast,
E~US coast, and S~Japan, i.e.~most of the coasts in
this thesis.

There are insufficient TCs in CMIP models~\cite{camargo2013global},
because the processes involved
in cyclogenesis are hard to resolve in climate models.
As is shown in Figure 19 of~\cite{williams2018met},
 for HadGEM3 this is particularly a problem in the North Atlantic~\cite{tomassini2017interaction}. % §~\ref{sec:cyclogenesis}.
CMIP models are colder in the tropics than observations~\cite{camargo2013global}.
For a recently published explanation see Seager et al.~2019~\cite{seager2019strengthening}.
This may lead to less TCs~\cite{tomassini2017interaction}.

TC PI theory is well supported,
but a few super-intense tropical cyclones have been observed~\cite{camargo2019tropical}.
This could be a result of treating the parameters $C_k$, $C_D$, as constant in the derivation (§~\ref{sec:param}).

The sea surface does not reach its steady state as soon as sea surface
stress is applied, which this leads
to the regression being distorted by this hysteresis.
It depends significantly on the previous points.
It would therefore make sense to give the ML algorithm
a series of previous $\eta$ values for different lags.


%\subsection{Summary of systematic errors}
\label{sec:sys-errors}
Table~\ref{tab:syst_error} shows the list of systematic errors that effect the
observed hazard, most of which will lead to an underestimate.

\begin{table}[ht]
\resizebox{\columnwidth}{!}{%
\begin{tabular}{l L{8.5cm}}
%\hline \hline
\textbf{Effect} & \textbf{Source}\\
\hline
($-$) & Coarsening of time to daily outputs.\\
($-$) & Lack of cyclogenesis  $\implies$ not enough
        TCs~\cite{tomassini2017interaction, williams2018met, FurtherInfo}.\\
($\pm$) & Parameterisation of heat transfer $\implies$ wrong TC strength.\\
($\pm$) & Parameterisation of wind stress $\implies$ wrong TC surge.\\
($-$) & SSH 4km out from the coast (centre of cell).\\
($-$) & No non-linear surge-tide interaction.\\
($-$) & Limited wave build up.\\
($-$) & Long TC return periods $> 100$~yrs (e.g.~New England).\\
($\pm$) & Inaccurate bathymetry.\\
\hline
\end{tabular}
}
\caption{Systematic sources of error for the estimate of TC-induced storm surge
hazard.}
\label{tab:syst_error}
\end{table}

