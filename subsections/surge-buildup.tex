\subsection{Surge buildup}

The majority of the surge is caused by the wind. rather than the low-pressure directly
(see Y20~\cite{ZannaPreprint} Figure 8).

Equation 10 in Y20~\cite{ZannaPreprint} (from Pugh, 1987~\cite{pugh1987tides} p.~89~\&~199) suggests that

\begin{equation}
\frac{\partial \eta}{\partial x} \approx \frac{\tau}{\rho_{0} g H},
\end{equation}

where $\eta$ is the dynamical sea level, $\rho_0$ is the seawater density,
$g$ is the gravitational acceleration, and $H$ is the ocean depth.
This implies that the surge buildup is greater the shallower the ocean $H$,
and therefore that coastlines with a shallowly growing coastal buildup will
be at greater risk of Storm surges.

\paragraph{Wave buildup}

A gradual slope will also favour the buildup of waves in the same location as due to
the better impedance matching, a greater percentage of the wave energy will be able to
meet the shore~\cite{pugh1987tides}. However those areas with shallowly sloping
bathymetry's may be protected by formidable headlands. Wave buildup is not included
in the models used here, and so we should expect the dominant effect to come from
a gradual bathymetric gradient.
