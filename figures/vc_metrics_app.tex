\begin{figure}[htb!]
\includegraphics[width=\linewidth]{../surge/plots/vc_bath_list.pdf}
\vspace{-25pt}

\caption{Isobaths plotted for \texttt{vc}.}
\label{fig:bath}
\includegraphics[width=\linewidth]{../surge/plots/vc_distance_isobath.pdf}
\vspace{-25pt}

\caption{Distance to isobaths from points on \texttt{vc}.}
\label{fig:vc_isobath}
\includegraphics[width=\linewidth]{../surge/plots/vc_isobath_correlate.pdf}
\vspace{-25pt}

\caption{The correlation matrix between the different isobath's distance's to
\texttt{vc}.}
\label{fig:vc_isobath}
\end{figure}


\begin{figure}[htb!]
\centering
\includegraphics[width=\linewidth]{../surge/plots/vc_angle_heatmap.pdf}
\caption{Normal bearing, $B^{\prime}$, along \texttt{vc}.

         }
\label{fig:angle_heatmap}

\includegraphics[width=\linewidth]{../surge/plots/vc_derivative_heatmap.pdf}
\caption{Convexity metric along \texttt{vc}.
         The three panels above show bay like concavity in blue, and convexity
         in red. At the largest $\sigma$ only the largest headlands visible (C), whereas the smallest bays are visible at a lower
        $\sigma$ (A).
}
\label{fig:derivative}
\end{figure}
