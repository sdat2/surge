\documentclass[../main.tex]{subfiles}
\begin{document}

\twocolumn[
  \begin{@twocolumnfalse}
  \maketitle
  \setcounter{page}{1}
    \begin{abstract}
     A Storm surge is the rapid, anomalous increase in coastal sea level that can
      occur during a storm, typically rising high above the local astronomical tide.
       Storm surge events can damage critical infrastructure, hamper economic activity,
        and endanger coastal communities. In the coming decades, these damaging events
         are expected to increase in severity and frequency due to sea level rise,
          which means that improving predictive skill in this area now is crucial.
           Despite the serious risks that they pose, our ability to understand storm
            surge risk as a function of local and remote variables is limited by
             technical issues (e.g. the computational expense and limited resolution
              of coupled air-sea numerical models, biases in model parameterisation schemes,
               a lack of resolved tides in large-scale model data). In order to address this gap
                in our knowledge, we will carry out numerical simulations and regression analysis
                 to help develop a functional relationship between local and remote variables
                  (e.g. bathymetry, sea level pressure, along-shore winds, offshore winds)
                   and storm surge risk. Specifically, we will use a numerical representation
                    of near-shore flow to develop hypotheses about the relationship between some "input"
                     variables and storm surge risk, as measured by a function of coastal sea surface
                      height. In addition, we will deploy machine learning
                       (e.g. Gaussian process regression, neural networks) on existing high-resolution,
                        coupled air-sea numerical simulation data to examine the factors that control
                         local storm surge risk. This data is from the Met Office ORCA12 runs, stored
                          on the JASMIN HPC platform. This project may lead to further work on
                           the storm surge likelihood to local economic and safety risks,
                            towards a model that directly links physical variables and human impact.
\paragraph{Keywords:} \textit{Gaussian Processes, Neural Network Emulators}
    \end{abstract}
    \vspace{20pt}
  \end{@twocolumnfalse}
]

\end{document}
