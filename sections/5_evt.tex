\section{Extreme value theory}
\label{sec:evt}
\subsection{Idealised statistical theory}
The central limit theorem is a well known triumph of classical statistics,
and it is often useful in the physical sciences where the central value
is that of interest. However, by its nature the mean SSH will
not be able to damage the coast (barring significant sea level rise that is not mitigated),
and life and property will only be put
at risk by the most extreme events.

The extreme value theorem is:


Thanks to Mood 1950~\cite{mood1950introduction}.
Let $X_{1}, X_{2} \ldots, X_{n} \ldots$ be a sequence of independent and
identically-distributed random variables, and $M_n = \max\{X_1, \ldots X_n\}$.
For a sequence of real numbers $(a_n, b_n) \exists: a_n>0$, and
$\lim _{n \rightarrow \infty} P\left(\frac{M_{n}-b_{n}}{a_{n}} \leq x\right)=F(x)$
where $F$ is a non-degenerate distribution function, then the limit distribution
F belongs to either the Gumbel, the Frechet, or the Weibull family,
but as T19~\cite{taleb2019statistical} states, `life is lived in the pre-asymptotics'.


Extreme value theory is the provides the extension of this theory
to look at the return period of extreme events of a certain value.

Both require domain of attraction condition~\cite{bucher2018horse}:

\begin{enumerate}
  \item Either $\forall r \in \mathbb{N} \;\exists \;b_r, \gamma\in \mathbb{R},\; a_r>0: $
    \begin{eqnarray}
    \lim _{r \rightarrow \infty} F^{r}\left(a_{r} x+b_{r}\right)=\exp \left\{-(1+\gamma x)^{-1 / \gamma}\right\} \notag \\
     \text { for all } 1+\gamma x>0
    \tag{BM}
    \end{eqnarray}

  \item Or equivalently, there exists a positive function $\psi=\psi (t):$
    \begin{eqnarray}
    \lim _{t \uparrow x^{*}} \frac{1-F(t+\psi(t) x)}{1-F(t)}=(1+\gamma x)^{-1 / \gamma} \notag \\
    \text { for all } 1+\gamma x>0
    \tag{POT}
    \end{eqnarray}
 \end{enumerate}

Normal Choices are POT or Block Maxima, but these strategies require
large amounts of data to converge~\cite{taleb2019much}.

BM leads to Generalised Extreme Value (GEV) distribution,
      whereas POT leads to Generalised Pareto Distribution (GPD).

\subsection{Block maxima GEV}
It is trivial to extract the highest SSH value in a given year at a given point
from \texttt{control-1950}.
It was decided not to subtract the low frequency SSH seasonal cycle from the
values first, because it is the absolute height of the sea surface which
creates the hazard rather than the relative height.

\begin{figure*}[htb!]
    \centering
    \vspace{-25pt}
    \includegraphics[width=0.8\linewidth]{../surge/plots/GEV_modelNO.pdf}
    \vspace{-15pt}
   \caption{New Orleans GEV plot adapted from \texttt{skextremes}~\cite{skextremes}
            for \texttt{c50} with $\alpha=0.05$.
            The deviation from the fitted distribution
            is shown in A~\&~C
            - does this represent TCs? (see Figure~\ref{fig:return_hyp_new}). The parameters above are quoted with 1$\sigma$ error bars.
            C's error bars are 2$\sigma$.}
    \label{fig:gev-no}
    \includegraphics[width=0.8\linewidth]{../surge/plots/skextreme_second_tactic.pdf}
    \vspace{-15pt}
   \caption{GEV parameters for \texttt{c50}, \texttt{eUS}.
   The confidence intervals, particularly in $\xi$, show that the GEV MLE fit of the
   is relatively poor. $\xi$ is more likely to be negative
   in the Gulf of Mexico, which could because of higher
   TC activity along this coastline (Figure~\ref{fig:top-100}). The $r_p$ above
   show that $\mu$ and $\sigma$ are highly correlated (all $r_p$ have 1\%$\gg$p)
   $\mu$ and $\sigma$ are also correlated with the responsiveness in §~\ref{sec:responsiveness}
   (for $\mu$: $r_p=0.44\pm0.02$, p$<10^{-10}$).}
    \label{fig:gev_all_points}s
\end{figure*}


\begin{figure}[htb!]
    \centering
    \includegraphics[width=1\linewidth]{images/taleb-limit-slimmed.png}\\
    \textit{Figure 15.1 from T19~\cite{taleb2019statistical} p.~279}
   \caption{As shown if you only
   observe a distribution up to some value M,
    you may be tempted to fit a line through the
   data (dotted blue line).
   But if there were in fact a limit to the distribution at H,
   you would be overestimating
   the true number of very extreme events (red curve)
   and also predict events that were larger than were possible.}
   \label{fig:up-bound-taleb}

\end{figure}


\subsection{Comparison to POT GPD}
The other alternative is to use the peak-over-threshold method.
There are a number of alternatives as to how you should choose the
threshold.

\subsection{Enforcing an asymptote using potential intensity theory }
As noted in~§~\ref{sec:hurr-theory} there is some maximum size
that a tropical cyclone might be expected to be able to reach given
the climate. This gives us information as to the shape of the probability
distribution beyond what just curve-fitting the tail.


\begin{figure}[htb!]
    \centering
    \includegraphics[width=1\linewidth]{images/Return_Hypothesis.pdf}
    \vspace{-15pt}
   \caption{The maximum height is a function of the potential intensity
   allowed by the climate, and the responsiveness of that point on the coastline to a
   wind stress of that size. If T$_0$ is a similar or greater than the time period of
   measurement, then it is possible that no hurricanes exist in the data sample.
   Absence of evidence is not evidence of absence however, and some areas of the US
   coastline such as Galverston and New England experience hurricanes infrequently,
   but as respectively in 1900 and 1908, when they do come they can cause great devastation~\cite{emanuel2005divine}.
   The low TC frequency bias in CMIP models will exacerbate this problem. This will
   appear as the blue curve continued upwards. The period of time
   between T$_0$ and T$_1$ likely depends on the average size of the tropical cyclone storms, and the
   percentage of them which rise to their maximum severity.
   For highly irresponsive places (e.g. Miami), it is possible that the deviation caused
   by hurricane landfall does not cause a large deviation from the existing EV distribution,
   that is mainly caused by other factors (the Florida current).
   }
   \label{fig:return_hyp}

\end{figure}

